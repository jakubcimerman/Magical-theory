\documentclass[../../main.tex]{subfiles}
% 
\begin{document}
\chapter{Častice a ich vzájomné interakcie}
\section{Zadanie}
Častice a ich vzájomné interakcie, Interakcie medzi elementárnymi časticami, Vlastnosti elementárnych častíc, Klasifikácia elementárnych častíc, Hadróny, Leptóny, Antičastice, Symetrie a zákony zachovania, Štandardný model, Zákony zachovania energie a hybnosti, Súradnicové sústavy v subjadrovej fyzike, Transformácie kinematických veličín medzi sústavami, Mandelstamové premenné, Kinematické premenné – rapidita. pseudorapidita, Feynmanova premenná, Bjorkenova premenná

%OOOOOOOOOOOOOOOOOOOOOOOOOOOOOOOOOOOOOOOOOOOOO
%OOOOOOOOOOOOOOOOOOOOOOOOOOOOOOOOOOOOOOOOOOOOO
%OOOOOOOOOOOOOOOOOOOOOOOOOOOOOOOOOOOOOOOOOOOOO
\section{Štandardný model}
\subsection{História}
V roku 1960 navrhol Sheldon Glashow teoretickú možnosť ako skombinovať elektromagnetickú a slabú interakciu do jednotnej teórie. O sedem rokov neskôr doplnili Steven Weinberg a Abdus Salam navrhnutý teoretický model o Higgsov mechanizmus, ktorý priamo determinuje hmotnosti elementárnych častíc popísaných v rámci štandardného modelu. Špeciálne ide hlavne o hmotnosti W a Z bozónov a fermiónov. Higgsov mechanizmus takisto vysvetľuje, akým spôsobom získavajú hmotnosť kvarky a leptóny.

Po objave slabých neutrálnych prúdov v CERN-e, spôsobených výmenou Z bozónov sa elektroslabá teória stala široko akceptovanou. Glashow, Salam a Weinberg, tvorcovia tejto teórie, následne dostali v roku 1979 Nobelovú cenu za fyziku. Neskôr, v 1981 boli experimentálne objavené bozóny W a Z. Experimentálne boli určené ich hmotnosti, pričom tieto hmotnosti boli v dobrej zhode s predpoveďami poskytnutými Štandardným modelom. Teória silnej interakcie získala svoju modernú podobu v 70. rokoch, kedy experimenty potvrdili, že hadróny sú zložené z nabitých kvarkov.

\subsection{Prehľad}
Štandardný model fyziky častíc je zjednotený súbor teoretických poznatkov zahrňujúci väčšinu známych elementárnych častíc. V rámci modelu je možné zjednoteným spôsobom ( zjednotenou matematickou formuláciou ) popísať tri zo štyroch fundamentálnych interakcií: silnú, slabú, a elektromagnetickú. Štandardný model predstavuje relativistickú kvantovú teóriu vyhovujúcu zároveň princípom špeciálnej teórie relativity i kvantovej mechaniky. Gravitačná interakcia a teda ani všeobecná teória relativity nie sú v modeli zahrnuté. Fundamentálnymi objektmi vystupujúcimi v tejto teórii sú polia v časopriestore. Štandardný model bol vypracovávaný postupne. Jeho základy boli položené začiatkom 20. storočia. Súčasná formulácia bola dokončená po experimentálnom potvrdení existencie kvarkov. Táto teória je v dobrom súlade so súčasnými experimentálnymi údajmi. Zahrňuje však 18 voľných parametrov, ktorých hodnotu nepredpovedá. Hodnota týchto parametrov je určená výhradne na základe experimentálnych výsledkov. Nepopisuje taktiež gravitáciu, tmavú hmotu či tmavú energiu.
 
Štandardný model je kalibračná teória silných (SU(3)) a elektroslabých (SU(2) $\times$ U(1)) interakcií s kalibračnou grupou nazývanou tiež Štandardný model symetrickej grupy SU(3) $\times$ SU(2) $\times$ U(1). V nasledujúcej časti si povieme o elementárnych časticiach a interakciách medzi nimi.

\section{Elementárne častice a ich klasifikácia} 
Pod pojmom elementárna častica alebo fundamentálna častica rozumieme časticu, ktorej subštruktúra je neznáma, a teda nie je známe, či je daná častica zložená z  menších častíc. Tieto častice môžu byť rozdelené do dvoch skupín: elementárne fermióny a bosóny.
\subsection{Elementárne fermióny} 
Do tejto skupiny patria kvarky a leptóny, ktoré tvoria hmotu okolo nás preto ich môžme nazvať aj časticami hmoty. Rozdelenie týchto častíc je zobrazené v tabuľke (\ref{sf1:fig:Tabulka_fermionov}).
\begin{figure}[!h]
\includegraphics[width=1.0\textwidth]{Vlastnosti_tabulka.png}
\caption{Tabuľka fermiónov.}
\label{sf1:fig:Tabulka_fermionov}
\end{figure}

Všetky elementárne fermióny sú častice so spinom (1/2), antisymetrickou vlnovou funkciou, spĺňajúce Pauliho vylučovací princíp a ich správanie určuje Fermi-Diracovo rozdelenie, ktoré vyzerá nasledovne
\begin{equation}
f(\epsilon_i)=\frac{1}{e^{(\epsilon_i-\mu)/kT}+1},
\end{equation}
kde $k$ je Boltzmanová konštanta, $T$ je absolútna teplota, $\epsilon_i$ je energia jedno-časticového stavu $i$ a $\mu$ je celkový chemický potenciál.

Nabité \textbf{leptóny} ($e^{-}$, $\mu^{-}$, $\tau^{-}$) interagujú elektromagnetickou a slabou interakciou, zatiaľ čo neutrálne leptóny (neutrína) interagujú iba slabou interakciou. Leptóny neinteragujú silnou interakciou!

\textbf{Kvarky} interagujú silnou, slabou a elektromagnetickou interakciou. Každý kvark nesie jeden z troch farebných nábojov silnej interakcie (green, red, blue). Izolované kvarky neboli nikdy v prírode pozorované a vyskytujú sa len vo viazaných stavoch zvaných \textbf{hadróny}, ktoré majú neutrálny farebný náboj. Niektoré hadrónov sú takmer stabilné a niektoré (známe ako rezonancie) majú extrémne krátku životnosť. Stupeň stability závisí hlavne od hmotnosti hadrónu. Hadróny môžme rozdeliť na baryóny a mezóny. 

\begin{itemize}
\item \textbf{Baryóny} sú zložené častice, ktoré obsahujú 3 kvarky a majú polovičný spin, napr.(proton-uud, neutron-udd, $\Lambda$-uds). Baryón, ktorý obsahuje jeden alebo viac strange kvarkov, ale žiadny charm, bottom alebo top kvark, sa nazýva hyperón. Keďže silné interakcie si zachovávajú zvláštnosť (strangeness), hyperóny sa nemôžu rozpadnúť silnou interakciou avšak zučasťnúju sa silnej interakcie (to znamená, že môžu vzniknúť silnou interakciou). Rozpadajú sa niekoľko-násobnou slabou interakciou, ktorá mení ich strangeness, poväčšine na protón alebo neutrón (slabá interakcia podivnosť nezachováva).
\item \textbf{Mezóny} sú zložene z jedného kvarku a jedného antikvarku a výsledný mezón musí byť bezfarebný. Všetky mezóny sú nestabilné, pričom najdlhšia životnosť trvá len niekoľko stotín mikrosekúnd. Nabité mezóny sa rozpadajú na elektróny a neutrína (ako môžu sa rozpadnúť aj na iné mezóny ale tie sa potom tiež rozpadnú až to väčšinou skončí na leptonóch). Nenabité mezóny sa môžu rozpadnúť aj na fotóny. Obe tieto rozpady naznačujú, že farebný náboj už nie je vlastnosťou vedľajších produktov. Rozlišujú sa mezóny skalárne (spiny kvarku a antikvarku sú orientovane opačne, takže výsledný spin mezónu je 0) a mezóny vektorové (spin kvarku a antikvarku majú rovnaký smer, takže výsledný spin mezónu je 1). Mezóny sa zaraďujú medzi bozóny, keďže majú celočíselný spin avšak, nie medzi elementárne bozóny. 
\end{itemize}
\subsection{Elementárne bozóny}
Sú to častice, ktoré sprostredkúvajú základne interakcie; fotón pre elektromagnetickú interakciu, bozóny $W^{\pm}$, $Z^0$ pre slabú interakciu a gluóny pre silnú interakciu. Tieto častice majú celočíselný spin, symetrickú vlnovú funkciu, nespĺňajú Pauliho vylučovací princíp a ich správanie je riadene Bose-Einsteinovou štatistikou, ktorá ma nasledujúci tvar
\begin{equation}
f(\epsilon_i)=\frac{1}{e^{(\epsilon_i-\mu)/kT}-1}.
\end{equation}
Do tejto skupiny patrí aj Higgsov bozón, ktorý ma nulový spin.

\section{Fundamentálne interakcie}
Ešte než prejdeme k jednotlivému popisu jednotlivých interakcii, uvedieme tabuľku, v ktorej sú základne charakteristiky fundamentálnych interakcii, viď tabuľku \ref{sf1:fig:Tabulka_interakcie}
\begin{figure}[!h]
\includegraphics[width=1.0\textwidth]{Tabulka_interakcie.png}
\caption{Tabuľka interakcii.}
\label{sf1:fig:Tabulka_interakcie}
\end{figure}
\subsection{Elektromagnetická interakcia}
Prvou interakciou, ktorou sa budeme zaoberať, je elektromagnetická interakcia, ktorá pôsobí medzi časticami s nenulovým elektrickým nábojom. Mediátorom tejto interakcie je fotón, čo je vektorová častica (spin = 1).

Opis tejto interakcie začneme najprv z klasického hľadiska. V klasickom elektromagnetizme sa elektromagnetické pole riadi známymi Maxwellovými rovnicami
\begin{equation}
\begin{gathered}
\nabla \cdot \vec{E} = \frac{\rho}{\epsilon_0}\\
\nabla \cdot \vec{B} = 0\\
\nabla \times \vec{E}=-\frac{\partial \vec{B}}{\partial t}\\
\nabla \times \vec{B}= \mu_0 \vec{j}+\mu_0\epsilon_0\frac{\partial \vec{E}}{\partial t}
\end{gathered}
\end{equation} 
Prvá rovnica opisuje, ako sú elektrické polia vyvolané nábojmi. Druhá rovnica hovorí, že neexistuje nič také ako magnetický monopol. Tretia rovnica opisuje indukciu elektrických polí zmenou magnetických polí a štvrtá rovnica opisuje generovanie magnetických polí elektrickými prúdmi a indukciu magnetických polí časovou zmenou elektrických polí.

Z druhej a tretej Maxwellovej rovnice sa navyše dá ukázať, že polia $\vec{E}$ a $\vec{B}$ môžu byť prepísané následovne 
\begin{equation}
\begin{gathered}
\vec{B} = \nabla \times \vec{A}\\
\vec{E} = -\nabla\varphi-\frac{\partial}{\partial t}\vec{A},
\end{gathered}
\end{equation}  
kde funkcia $\vec{A}$ sa nazýva vektorový elektromagnetický potenciál a funkcia $\varphi$ je skalárny elektromagnetický potenciál. Skalárny a vektorový potenciál sú určené hustotami elektrického náboja a prúdu prostredníctvom rovníc, ktoré dostaneme zo zvyšných dvoch Maxwellových rovníc, keď do nich dosadíme $\vec{B}$ a $\vec{E}$ vyjadrené cez dane potenciály. Čo je však dôležitejšie je to, že tieto potenciály nie sú určene jednoznačne. A tak môžme z potenciálu $\vec{A}$ prejsť na potenciál 
\begin{equation}
\vec{A} \rightarrow \vec{A}+grad(\Lambda).
\end{equation}
Toto môžme urobiť preto lebo rotácia gradientu akejkoľvek vektorovej funkcie je vždy nula, takže naše $\vec{B}$ sa v konečnom dôsledku vôbec nezmení. Následne aby sa nezmenil ani skalárny potenciál tak aj ten musí prejsť z $\varphi$ na
\begin{equation}
\varphi \rightarrow \varphi -\frac{\partial}{\partial t}\Lambda.
\end{equation}
Touto transformáciou poli sa nezmení ani $\vec{B}$ ani $\vec{E}$. Uvedená transformácia elektromagnetických potenciálov (oboch súčasne!) sa nazýva \textbf{kalibračná transformácia}.

Prečo sme to ale vlastne cele robili a zaviedli sme takéto potenciály? Odpoveďou napríklad je, že kvantová mechanika častice v elektromagnetickom poli je opísaná Schrodingerovou rovnicou, v ktorej vystupujú elektromagnetické potenciály a nie elektromagnetické polia. Nahradenie potenciálov poliami by tu bolo značne komplikovane a neprirodzene. Navyše kvantová teória samotného elektromagnetického poľa, tzv. kvantová elektrodynamika, je založená na tzv. kvantovaní klasickej teórie. K tomuto kvantovaniu je potrebne mat sformulovanú klasickú elektrodynamiku v lagrangeovskom alebo hamiltonovskom formalizme. Pre oba tieto formalizmy su elektromagnetické potenciály oveľa vhodnejšie a prirodzenejšie ako elektromagnetické polia.

Problémy klasického elektromagnetizmu nastali keď Einstein publikoval teóriu fotoelektrického javu, v ktorej predpokladá, že svetlo se nešíri ako vlnenie elektromagnetického poľa, ale môže existovať vo forme častíc, diskrétnych kvánt, neskor nazývaných fotóny. Einsteinová teória fotoelektrického javu bola v súlade s predstavami, ktoré sa objavili v navrhnutom riešení Maxa Plancka v roku 1900. Vo svojej práci Planck predpokladal, že elektromagnetické vyžarovanie telies prebieha cez diskrétne kvantá, čo vedie ku konečnej celkovej energii. Tato predstava bola v priamom protiklade s klasickým pohľadom na svetlo ako spojitú vlnu. Plancková a Einsteinová teória následne viedla ku kvantovej mechanike, ktorá bola formulovaná v roku 1925. Na jej základe bola okolo roku 1940 dokončená nová kvantovo-mechanická teória elektromagnetizmu; kvantová elektrodynamika (QED) a je jednou z najpresnejších fyzikálnych teórii.

\textbf{Kvantová elektrodynamika} je náuka o pohybe elektrických nábojov (nabitých telies) v obecne premenných elektromagnetických poliach. Klasická elektrodynamika študuje elektrodynamické interakcie medzi makroskopickými telesami, kvantová elektrodynamika interakcie medzi mikro-objektmi. QED popisuje interakciu žiarenia s hmotou (fotoelektrický jav, Comptonov rozptyl, brzdné žiarenie), elektromagnetické interakcie medzi nabitými elementárnymi časticami prostredníctvom fotónov. Kvantová elektrodynamika vznikla ako teória interakcie elektromagnetického poľa a poľa popisujúceho elektróny a pozitróny.

A poďme teraz na samotný matematicky aparát QED. Tento bude trochu dlhší ako tie ďalšie dva a to len preto aby sme si demonštrovali silu už spomenutých kalibračných transformácii. Začneme veľmi z ľahká a to tým, že si napíšeme Diracov lagrangian pre diracovú voľnú časticu (kde $c$=$h$=1).
\begin{equation}
\mathcal{L}_D=\bar{\psi}(i\gamma^{\mu}\partial_{\mu}-m)\psi
\end{equation}
Pomocou Euler–Lagrangeovej rovnice pohybu pre pole, ktorá ma tvar
\begin{equation}
\partial_{\mu}\bigg(\frac{\partial \mathcal{L}}{\partial(\partial_{\mu}\psi)}\bigg)-\frac{\partial\mathcal{L}}{\partial\psi}=0,
\end{equation}
sme schopný dostať Diracovú rovnicu v tvare
\begin{equation}
(i\gamma^{\mu}\partial_{\mu}-m)\psi=0.
\end{equation}
Toto je pohybová rovnica pre voľné elektróny. V prípade pozitrónu by sme dostali
\begin{equation}
\bar{\psi}(i\gamma^{\mu}\partial_{\mu}+m)=0.
\end{equation}
Z týchto dvoch rovníc (keď ich sčítame a vynásobíme $\bar{\psi}$, $\psi$) môžme odvodiť rovnicu kontinuity pre 4-vektor prúdu
\begin{equation}
\partial_{\mu}j^{\mu}=0,
\end{equation}
kde $j=e\bar{\psi}\gamma^{\mu}\psi$.
Toto odvodenie bolo na klasickej hladine, v kvantovom prípade by to bolo úplne to iste len akurát by to musela byt normálne usporiadaná nábojová hustota. Pre väčšie detaily ohľadom tohto usporiadania pozri ($http://sophia.dtp.fmph.uniba.sk/~peterp/QED_A.pdf$).

Ako som už spomínal toto odvodenie bolo len pre voľnú diracovú časticu, ktorá s ničím neinteragovala. Teraz však budeme chcieť aby s naším nabitým poľom $\psi$ interagovalo nejaké ďalšie pole. Ako ale pridať nejaké ďalšie pole tak aby sme nenarušili celu tuto konštrukciu? Môžme si všimnúť, že fyzikálne veličiny ako hustota náboja ($\bar{\psi}\psi$) alebo prúd ($\bar{\psi}\gamma^{\mu}$) sú invariantné ak pridáme lokálnu fázu $\Lambda(x)$ do pola $\psi$.
\begin{equation}
\begin{gathered}
\psi(x)\rightarrow e^{iq\Lambda(x)}\psi(x)\\,
\bar{\psi(x)}\rightarrow e^{-iq\Lambda(x)}\bar{\psi(x)},
\end{gathered}
\end{equation}
táto transformácia sa nazýva lokálna U(1) kalibračná transformácia. Kebyže tuto transformáciu aplikujeme na člen $\bar{\psi}\partial_{\mu}\psi$ zistíme, že tento člen nie je invariantný pre tuto transformáciu pretože derivácia ($\partial_{\mu}$) sa pod touto U(1) symetriou netransformuje invariantné.
\begin{equation}
\bar{\psi}\partial_{\mu}\psi \rightarrow (\bar{\psi}e^{-iq\Lambda(x)})\partial_{\mu}(e^{iq\Lambda(x)}\psi)=\bar{\psi}(\partial_{\mu}+iq\Lambda(x))\psi\neq \bar{\psi}\partial_{\mu}\psi.
\end{equation}
Aby sme vyriešili nekovariantnosť derivácie a spravili tak lagrangian kalibračné invariantný, musíme zaviesť kalibračné pole $A_{\mu}$ a to následovne
\begin{equation}
D_{\mu}=\partial_{\mu}-iqA_{\mu},
\end{equation}
kde ako už vieme $A_{\mu}$ sa musí transformovať ako $A_{\mu}\rightarrow A_{\mu}+\partial_{\mu} \Lambda(x)$. $D_{\mu}$ sa nazýva kovariantná derivácia a je invariantná pod lokálnymi kalibračnými transformáciami, čo vlastne znamená
\begin{equation}
\bar{\psi}D_{\mu}\psi=\bar{\psi}(\partial_{\mu}-iqA_{\mu})\psi \rightarrow \bar{\psi}e^{-iq\Lambda(x)}(\partial_{\mu} - iq(A_{\mu}+\partial_{\mu}\Lambda(x)))e^{iq\Lambda(x)}\psi=\bar{\psi}D_{\mu}\psi.
\end{equation}
Keď teraz do lagrangianu pre voľnú časticu vložíme túto kovariantnú deriváciu namiesto normálnej parciálnej derivácie ($\partial_{\mu}\rightarrow D_{\mu}$) a vykonáme na ňom kalibračnú transformáciu všetkých polí
\begin{equation}
\begin{gathered}
\psi \rightarrow e^{iq\Lambda(x)}\psi, \\
\bar{\psi} \rightarrow e^{-iq\Lambda(x)}\bar{\psi}, \\
A_{\mu}\rightarrow A_{\mu}+\partial_{\mu}\Lambda(x),
\end{gathered}
\end{equation}
tak dostaneme lagrangian, ktorý môžme napísať v tvare
\begin{equation}
\mathcal{L}=\bar{\psi}(i\gamma^{\mu}\partial_{\mu}-m)\psi+q\bar{\psi}\gamma^{\mu}\psi A_{\mu}.
\end{equation}

Ako môžme vidieť, tento lagrangian pozostáva z pôvodného Diracovho lagrangianu pre voľnú časticu a nového interakčného členu medzi poľom častice a novým kalibračným poľom. Symbol $q$ značí elektrický náboj častice. Tento lagrangian už obsahuje to, čo sme chceli, akurát nie je kompletný a to z toho dôvodu, že mu chýba kinetický člen pre pole $A_{\mu}$. Tento člen sa dá ľahko dostať z Procovho lagrangianu, ten použijeme preto lebo pole $A_{\mu}$ musí reprezentovať vektorovú časticu
\begin{equation}
\mathcal{L}_{Proc}=-\frac{1}{4}F_{\mu\nu}F^{\mu\nu}+\frac{1}{2}m^2A_{\mu}A^{\mu}.
\end{equation}
Člen, ktorý obsahuje hmotnosť nie je kalibračné invariantný a preto položíme hmotnosť toho pola rovnú nule. Pole $A_{\mu}$ bude reprezentovať fotón. A teraz môžme písať lagrangian pre kvantovú elektrodynamiku
\begin{equation}
\mathcal{L}_{QED}=\bar{\psi}(i\gamma^{\mu}\partial_{\mu}-m)\psi+q\bar{\psi}\gamma^{\mu}\psi A_{\mu}-\frac{1}{4}F_{\mu\nu}F^{\mu\nu}.
\end{equation}
Vložením tohto lagrangianu do Euler-Lagrangeovej rovnice pohybu pre pole, dostaneme 
\begin{equation}
\begin{gathered}
(i\gamma^{\mu}\partial_{\mu}-m)\psi=q\gamma^{\mu}A_{\mu}\psi\\
\partial_{\mu}F^{\mu\nu}=q\bar{\psi}\gamma^{\nu}\psi=qj^{\nu}
\end{gathered}
\end{equation}
Prvá rovnica je Diracová rovnica pre časticu v elektromagnetickom poli a druhá rovnica je súbor Maxwellových rovníc so zdrojom $j^{\nu}$, ktorý pochádza z Diracovej rovnice.

Povedzme si teraz nejaké vlastnosti a výsledky QED. Veľkosť tejto interakcie je charakterizovaná konštantou jemnej štruktúry
\begin{equation}
\alpha=\frac{1}{4\pi \epsilon_0}\frac{e^2}{\hbar c}
\end{equation}
Grafickou reprezentáciou procesov QED sú Feynmanove diagramy. Najčastejšie používane a najjednoduchšie sú diagramy na tkz. stromovej úrovni (three-level approximation), čo sú diagramy odpovedajúce prvému príspevku poruchovej teórie. Keďže QED je prototypom kvantovej teórie poľa je charakterizovaná dvomi dôležitými vlastnosťami: kalibračnou invarianciou, čo sme si už povedali a renormalizovateľnosťou.

Vo všetkých výpočtoch QED vystupujú divergentné členy. Aby sme im zabránili v divergovaní, bolo objavené, že je možné predefinovať hmotnosť a náboj. Akési, holé hmotnosti $m_0$ a náboje $e_0$ (nemerateľné hodnoty) je vždy možné prenásobiť bezrozmerným členom tak, aby sme dostali fyzikálne veličiny $m$ a $e$, ktoré už sú určené z experimentu. Ďalším dôležitým bodom pri renormalizáci je to, že väzbové konštanty (ako napr. $\alpha$) v skutočnosti nie sú konštantami, ale závisia na škále energie, na ktorých sa vykonávajú experimenty.

Jedným z najznámejších triumfov teórie kvantovej elektrodynamiky je presná predpoveď elektrónového faktora $g_s$, ktorý vystupuje v spinovom magnetickom dipólovom momente 
\begin{equation}
\vec{\mu}_s=-g_s\mu_B\frac{\vec{S}}{\hbar}.
\end{equation}
Z Diracovej rovnice vychádza, že $q_s=2$. Avšak, experimentálne sa ukázalo, že to nie je presne $2$ ale $2.00231930436182$. Vidíme, že tato hodnota je len o dvetisíciny väčšia ako hodnota vychádzajúca z Diracovej rovnice. Malá korekcia je známa ako \textit{anomálny magnetický dipólový moment elektrónu}. Vyplýva to z interakcie elektrónov s virtuálnymi fotónmi v kvantovej elektrodynamike.

\subsection{Slabá interakcia}
Slabá interakcia je mechanizmus interakcie medzi subatómovými časticami, ktorý spôsobuje rádio-aktívny rozpad. Tento mechanizmus môžme nazvať tkz. pomaly rozpad, pretože vznik a rozpad častíc pod v vplyvom silnej interakcie prebieha v časoch rádovo rovných alebo kratších ako $10^-{22}\,s$, zatiaľ čo doby života častíc rozpadajúcich sa pod vplyvom slabej interakcie sú omnoho kratšie než $10^-{13}\,s$. Najznámejším príkladom je $\beta$ rozpad neutrónu alebo miónu.
\begin{equation}
\begin{gathered}
n \rightarrow p\hspace{0.1cm}+\hspace{0.1cm}e^-\hspace{0.1cm}+\hspace{0.1cm} \bar{\nu}_e \hspace{0.3cm} \textit{s} \hspace{0.3cm} \tau \approx 881s \\ 
\mu^- \rightarrow e^- \hspace{0.1cm}+\hspace{0.1cm}\bar{\nu}_e\hspace{0.1cm}+\hspace{0.1cm} \nu_{\mu} \hspace{0.3cm} \textit{s} \hspace{0.3cm} \tau \approx 2.2\times10^-6s 
\end{gathered}
\end{equation}
Prvá teória $\beta$ rozpadu pochádzala od Fermiho a počítala so štvorfermiónovým vertexom
\begin{equation}
\mathcal{L}_{int}^{Fermi}=-G(\bar{\psi}_p\gamma^{\mu}\psi_n)(\bar{\psi}_e\gamma_{\mu}\psi_{\bar{\nu}})+h.c.
\end{equation}
Avšak ukázalo sa, že pri beta premene môže dochádzať k procesom, v ktorých sa mení spin (Gamow-Teller prechod). Následne ešte niekoľko experimentov ukázalo, že dochádza k narušeniu parity. Fermiho lagrangian niečo také nemal v sebe. Preto bolo potrebné vymyslieť niečo, čo bude v súlade s experimentálnymi pozorovaniami. Po zobratí do úvahy vtedajších výsledkov nadobudol interakčný lagrangian takýto tvar
\begin{equation}
\mathcal{L}^{\beta}_{int}=-\frac{G_{\beta}}{\sqrt{2}}\big[ \bar{\psi}_p\gamma_{\mu}(1-\gamma_5)\psi_n \big] \big[ \bar{\psi}_e\gamma^{\mu}(1-\gamma_5)\psi_{\nu} \big]+h.c.,
\end{equation}
kde $G_{\beta}=1.136 \times 10^{-5}\,GeV^{-2}$. Tento lagrangian už v sebe ma zakódované to, že slabá interakcia podlieha celkovému narušeniu parity. Pre elektróny to znamená, že sú takmer všetky ľavo-točivé a anti-neutrína sú naopak pravo-točivé.

Približne v tom čase, keď vznikala táto teória bol objavený mión, ktorý bolo môžme popísať takýmto lagrangianom
\begin{equation}
\mathcal{L}^{\mu}_{int}=-\frac{G_{\mu}}{\sqrt{2}}\big[ \bar{\psi}_{\nu_{\mu}}\gamma_{\alpha}(1-\gamma_5)\psi_{\mu} \big] \big[ \bar{\psi}_e\gamma^{\alpha}(1-\gamma_5)\psi_{\nu_{e}} \big]+h.c.,
\end{equation}
kde $G_{\mu}=1.16639\times 10^{-5} GeV^{-2}$. Vidíme, že pre rôzne rozpady častíc, ktorých polčasy rozpadu sú veľmi odlišné, hodnoty väzbových konštánt $G_{\beta}$ a $G_{\mu}$ sú veľmi podobné. Táto skutočnosť viedla k myšlienke, že procesy nukleónov s leptónmy a leptonov so sebou samých sú riadené rovnakou silou (Tiomno-Wheeler triangle). A tak vznika teória od Feynmana a Gell-Manna, ktorá bola veľmi dôležitá vo vývoji slabej interakcie a ktorá bola v tvare tkz. current-current forme
\begin{equation}
\mathcal{L}^{w}_{int}=-\frac{G_F}{\sqrt{2}}J^{\rho}J^{+}_{\rho},
\end{equation} 
kde $G_F = G_{\mu}$ a prúd $J_{\rho}$ pozostáva z leptónovej a hadrónovej časti
\begin{equation}
J_{\rho}=\bar{\psi}_{\nu_{e}}\gamma_{\rho}(1-\gamma_5)\psi_{e} + \bar{\psi}_{\nu_{\mu}}\gamma_{\rho}(1-\gamma_5)\psi_{\mu}+J_{\rho}^{hadron.}
\end{equation}
K tomu aby sme mohli previazať minimálne rozdiely medzi $G_F$ a 
$G_{\beta}$ zavedieme parametrizáciu cez. tkz Gabibbo uhol
\begin{equation}
\cos (\theta_C)=\frac{G_{\beta}}{G_{F}}=0.974
\end{equation}
V tomto štádiu sa takáto parametrizácia môže javiť trochu umelá, pretože nie je jasné, prečo by mal byť určitý uhol vhodný na opis jednoduchého faktu, že $G_{\beta}<G_F$. Ozajstná sila tejto parametrizácie sa prejavy keď sa budú uvažovať procesy pri ktorých dochádza k zmene podivnosti (strangeness). Hlavnou podstatou tohto uhlu je vyjadriť silu slabej interakcie pri procesoch, ktoré zachovávajú alebo nezachovávajú podivnosť. Ukazuje sa, že pre procesy, ktoré nezachovávajú podivnosť je tato sila rovná $G_F\sin(\theta_C)$, zatiaľ čo pre procesy, ktoré zachovávajú podivnosť, to je $G_F\cos(\theta_C)$. Vzhľadom k tomu, že uhol $\theta_C$ je číselne malý, možno usudzovať, že úloha Cabibbo uhla spočíva v potlačovaní slabých procesov, ktoré menia podivnosť v pomere k tým, ktoré zachovávajú podivnosť, avšak tieto procesy nie sú zakázané. Tieto poznatky boli z väčšej miere zistene z experimentov a preto sa zaviedli dve výberové empirické pravidlá, ktorými sa slabá interakcia riadi
\begin{itemize}
	\item Procesy, v ktorých sa zmenila podivnosť viac ako o jednotku, sú veľmi silno potlačené: $\Xi\rightarrow n+\pi^{-}$, kde $\Delta S=2$ a B.R. je $1.9\times 10^{-5}$
	\item Druhe pravidlo je $\Delta S = \Delta Q$, ktoré platí pre semileptónové rozpady. Majme všeobecný rozpad: $$
		h_i=h_f+lepton \hspace{0.1cm} pair.
		$$ 
	Platí $\Delta S = S(h_f)-S(h_i)$ a $\Delta Q = Q(h_f)-Q(h_i)$. Všimnime si, že tieto hodnoty nie sú v absolútnej hodnote. Dobrým príkladom je takýto rozpad:
	$$
		\Sigma^-\rightarrow n+e^-+\bar{\nu}_e \hspace{0.3cm} kde \hspace{0.3cm} \Delta S=\Delta Q = 1
	$$
	takže, tento rozpad je oveľa častejší ako napríklad rozpad 
	$$
		\Sigma^+\rightarrow n+e^++\nu_e \hspace{0.3cm} kde \hspace{0.3cm} \Delta S=1 \neq \Delta Q = -1
	$$
\end{itemize}
Tieto pravidla sa použili aj na tvorbu prvého tvaru hadronového prúdu, ktorý obsahoval zatiaľ len 3 kvarky, menovite u, d, s. Takže, keď zahrnieme všetky tieto myšlienky tak celkový prúd môžme písať ako
\begin{equation}
J_{\rho}=\bar{\psi}_{\nu_{e}}\gamma_{\rho}(1-\gamma_5)\psi_{e} + \bar{\psi}_{\nu_{\mu}}\gamma_{\rho}(1-\gamma_5)\psi_{\mu}+\bar{\psi}_{u}\gamma_{\rho}(1-\gamma_5)(\psi_{d}\cos(\theta_C)+\psi_s\sin(\theta_C)).
\end{equation}
Avšak, tento model mal problémy popisovať procesy, v ktorých vychádzali divergentne členy v rozptylových amplitúdach. Zdrojom všetkých ťažkostí, ktoré vznikajú v tomto Fermiho modely, je dimenzionalita príslušnej väzbovej konštanty $G_F$. A preto bolo zase potrebne nejako upraviť vtedajšiu teóriu aby zrušila tieto divergencie. Formálne sa dá zbaviť rozmernej väzbovej konštanty a to tak, ak sa pôvodná interakcia "prúd x prúd" nahradí spojením slabého prúdu $J_{\rho}$ s nejakým vektorovým poľom
\begin{equation}
\mathcal{L}_{int}^{w}=\frac{g}{2\sqrt{2}}(J_{\mu}W^{+\mu}+J_{\mu}^+W^{-\mu}). 
\end{equation}
Teraz je konštanta $g$ bezrozmerná, čo sme chceli. Pole $W_{\mu}$ musí byť komplexne, pretože je spojené s nabitým prúdom. Vektorové pole $W_{\mu}$ propaguje slabú interakciu fermiónov a preto $W^+$ a $W^-$ označujeme ako intermediálne bozóny slabej interakcie so spinom 1. Navyše vieme, že slabá interakcia je krátko dosahová, čo znamená, že tento $W^{\pm}$ bozón musí byť veľmi hmotný. Vertex je znázornený na obrázku \ref{sf1:fig:W_boson}.
\begin{figure}[!h]
\centering
\includegraphics[width=0.6\textwidth]{W_boson.png}
\caption{Všeobecný rozpad W bozónu na leptonový pár.}
\label{sf1:fig:W_boson}
\end{figure}
\newline
Porovnaním predošlej teórie s touto dostávame, že platí
\begin{equation}
\frac{g^2}{8M_W^2}=\frac{G_F}{\sqrt{2}}.
\end{equation}
Keďže, W bozóny sú nositeľmi elektrického náboja tak je s nimi možná aj elektromagnetická interakcia. Postup odvodenia interakcie W bozónu s fotónom si opíšeme len slovne.

Keďže W bozón je hmotná vektorová častica tak musí spĺňať správanie popísane Procovým lagrangianom. V tomto lagrangiane transformujeme polia a derivácie pomocou lokálnej kalibračnej transformácie presne tak isto ako v prípade elektromagnetickej interakcie. Keď sa obmedzíme na členy, ktorých dimenzia nebude vyššia ako 4 tak po par úpravách dostaneme, že interakčný lagrangian medzi W a fotónom ma tvar $\mathcal{L}^{em}=\mathcal{L}_{WW\gamma}+\mathcal{L}_{WW\gamma \gamma}$. Takže náš celkový interakčný lagrangian ma tvar
$$
\mathcal{L}^{ew}=\mathcal{L}^{w}+\mathcal{L}^{em} =\mathcal{L}_{CC}+\mathcal{L}^{em}_{fermion}+\mathcal{L}_{WW\gamma}+\mathcal{L}_{WW\gamma\gamma}.
$$
Aj keď bol tento model navrhnutý, aby sa zbavil predošlých divergencii z Fermiho modelu, pri spojení elektrickej a slabej interakcie nám vznikli procesy, v ktorých sa objavujú ďalšie divergentne členy.
 
Veľký progres vo vývoji prišiel, keď sa aplikovali poznatky plynúce zo štúdie Yang-Millsovej teórie, ktorá je založená na ne-Abelovskej kalibračnej symetrii. Ukázalo sa, že tato symetria môže zrušiť nejaké nežiaduce divergencie. Dôkladné odvodenie perturbatívnej renormalizácie, ktoré je založená na ne-Abeliovskej kalibračnej symetrii a ktoré navyše zahŕňa Higgsov mechanizmus pre generovanie hmoty, bolo odvodene Hooft-om v roku 1971. Rozhodujúcim momentom bol experimentálny objav slabého neutrálneho prúdu v roku 1973, ktorý jasne ukázal, že kalibračný model, ktorý zahŕňa neutrálny vektorový bozón, môže byť použitý na opis reálneho sveta. Tento model bol následne vylepšovaný až nakoniec dospel do tvaru, ktorý navrhli Weinberg, Salam a Glashow, zvaný ako štandardný model elektroslabej interakcie.

Tento model je založený na ne-Abelovskej SU(2) $\times$ U(1) kalibračnej grupe. Príslušnými kalibračnými bozónmy sú 3 W bozóny izospinu z SU(2) grupy ($W_1$, $W_2$, $W_3$) a B bozón slabého-náboja z U(1) grupy. Všetky tieto polia sú bezhmotné. Až ich vzájomná kombinácia bude dávať už nám známe $W^{\pm}, Z^0, \gamma$ bozóny. Hmotnosť týchto častíc (okrem $\gamma$), vyplíva zo spontánneho narušenia symetrie, ktoré je základom tkz. Higgsovho mechanizmu, ktorý je založený na existencii jednej skalárnej, neutrálnej častici so spinom rovný nule - Higgsov bozón. Výsledný lagrangian bude vo všeobecnom tvare následovný
\begin{equation}
\mathcal{L}_{ew}=\mathcal{L}_{K}+\mathcal{L}_{N}+\mathcal{L}_{C}+\mathcal{L}_{H}+\mathcal{L}_{HV}+\mathcal{L}_{WWV}+\mathcal{L}_{WWVV}+\mathcal{L}_{Y},
\end{equation}
kde $\mathcal{L}_{K}$ je kinetický člen pozostavajúci z kvadratických členov, ktoré zahŕňajú dynamické členy a hmotnostných členov, $\mathcal{L}_{N}$ a $\mathcal{L}_{C}$ sú členy, ktoré obsahujú neutrálny a nabitý prúd. Ich komponenty obsahujú interakcie medzi fermiónmy a bozónmy, $\mathcal{L}_{H}$ obsahuje interakčné Higgs three-point and Higgs four-point self interakcie, $\mathcal{L}_{HV}$ obsahuje interakcie Higgsa s W,Z bozónom, $\mathcal{L}_{WWV}$
obsahuje three-point self interakciu W,Z,$\gamma$ bozónov, $\mathcal{L}_{WWVV}$ obsahuje four-point self interakciu W,Z,$\gamma$ bozónov a $\mathcal{L}_{Y}$ obsahuje Yukawovsku interakciu medzi fermiónmy a Higgsových bozónom.

Povedzme si teraz nejaké vlastnosti a výsledky z daného lagrangianu. \textbf{Unification condition} - je vzťah, ktorý viaže väzbové konštanty slabej interakcie a elektromagnetizmu. Táto podmienka môže byť vyjadrená nasledovne 
$$
e=g\sin(\theta_W)=g^,\cos(\theta_W),
$$
kde $q$ je väzbová konštanta pre SU(2) grupu, $q^,$ je väzbová konštanta pre U(1) grupu a $\theta_W$ sa vola weak mixing uhol alebo Weinbergov uhol, ktorým spontánne narušenie symetrie rotuje pôvodne $W_3$ a $B$ vektorové kalibračné bozóny. Experimentálna hodnota tohto uhlu je $\sin^2(\theta_W)=0.222\pm0.006$. Využitím tohto uhla sa dajú dané kalibračné polia nakombinovať tak, že vzniknú $Z^0$ a $\gamma$ bozóny. Graficky sa celá unification condition dá znázorniť nasledovne, viď obrázok \ref{sf1:fig:unifi}.
\begin{figure}[!h]
\centering
\includegraphics[width=0.6\textwidth]{Unification_condition.png}
\caption{g-väzbová konštanta pre SU(2) grupu, $q^{,}$ je väzbová konštanta pre U(1) grupu.}
\label{sf1:fig:unifi}
\end{figure}

Hmotnosti $W^{\pm}$ a $Z^0$ sa dajú vyjadriť ako 
$$
m_W=\bigg(\sqrt{\frac{\pi \alpha}{G_F\sqrt{2}}}\bigg)\frac{1}{\sin(\theta_W)}=80,42\,GeV/c^2, \hspace{0.5cm} m_Z=\frac{m_W}{cos(\theta_W)}=91.18\,GeV/c^2
$$
Uveďme základne pravidla pre konštrukciu vertexov pre slabé interakcie. V každom vertexe musí byť zachovaný elektrický náboj, leptonové číslo a počet kvarkov. Keďže nabité $W^{\pm}$ bozóny menia náboj kvarku, v slabých vertexoch sa nezachovávajú vône kvarkov. Ich farebný náboj sa ale zachovaná, lebo W bozóny nie sú nositeľmi farebného náboja. Je potrebne tiež zmieniť, že slabé interakcie pôsobiace prostredníctvom W bozónov nemenia generáciu leptónov. Slabé interakcie zahrňujúce W bozón sa nazývajú interakcie nabitých prúdov, naopak slabé interakcie sprostredkúvané Z bozónom sa nazývaju interakcie neutrálnych prúdov. Pravidlá pre vertexy Zqq sú veľmi jednoduché - nemení sa v nich leptónová generácia, kvarková vôňa ani farebný náboj.

\textbf{Miešanie kvarkov, Cabbibov uhol, CKM matice}
Ako sme už spomínali pri odvodzovaní lagrangianu, v 60. rokoch minulého storočia sa ukázali experimenty kedy došlo k tomu, že sa nezachovávala podivnosť. Tie sú síce potlačené oproti tým, čo nemenia podivnosť ale aj tak existujú a to bolo treba vysvetliť a popísať. S popisom prišiel Gabibbo, ktorý si všimol pozoruhodné súvislosti medzi známymi slabými procesmi. Pre procesy, kde sa podivnosť nemení ma efektívna hadrónová konštanta hodnotu $G_F\cos(\theta_C)$, pre podivnosť meniace procesy má táto efektívna konštanta hodnotu $G_F\sin(\theta_C)$. Experimentálne sa určilo, že Gabibbov uhol ma hodnotu $\theta_C=13.04^{\circ}$. V rámci dvoj-generačného modelu (u, s, d, c kvarky) je možné také zmiešavanie popísať pomocou reálnych koeficientov, ktoré je možne súhrnne zapísať do tvaru matice
\[ U_C=
\begin{pmatrix}
    \cos(\theta_C) & \sin(\theta_C) \\
    -\sin(\theta_C) & \cos(\theta_C) \\
\end{pmatrix}=
\begin{pmatrix}
    U_{ud} & U_{us} \\
    U_{cd} & U_{cs} \\
\end{pmatrix}
\]
tato matica popisuje zmiešavanie kvarkov. Toto zmiešavanie sa dá napísať nasledovne 
\[
\begin{pmatrix}
    \bar{u},\bar{c} \\
\end{pmatrix}=
\begin{pmatrix}
    \cos(\theta_C) & \sin(\theta_C) \\
    -\sin(\theta_C) & \cos(\theta_C) \\
\end{pmatrix}
\begin{pmatrix}
    d \\
    s \\
\end{pmatrix}
\]
Pre tri generácie je zmiešavanie kvarkov vyjadrené pomocou Cabibbo-Kobayashi-Maskawovou maticou
\[ V_{CKM}=
\begin{pmatrix}
    V_{ud} & V_{us} & V_{ub} \\
    V_{cd} & V_{cs} & V_{cb} \\
    V_{td} & V_{ts} & V_{tb} \\
\end{pmatrix}
\]
jej elementy sú obecne komplexne (dajú sa parametrizovať pomocou troch uhlov Gabibbovho typu a jednou fázou). Presne vyjadrenie elementov CKM matice patri k hlavným a aktuálnym cieľom experimentálnej časticovej fyziky lebo predstavuje jeden zo zásadných testov správnosti Štandardného modelu elektroslabej interakcie.
\[
\begin{pmatrix}
    \bar{u} & \bar{c} & \bar{t} \\
\end{pmatrix} 
\begin{pmatrix}
    0.975 & 0.221 & 0.022 \\
    0.221 & 0.974 & 0.040 \\
    0.009 & 0.039 & 0.999 \\
\end{pmatrix}
\begin{pmatrix}
    d \\
    s \\
    b \\
\end{pmatrix}
\]
Hodnoty CKM matice na diagonále sú skoro rovnaké veľké a blízko jednotky, čo implikuje napríklad, že $t$ kvark sa s najväčšou pravdepodobnosťou rozpadne na $b$ kvark. Nediagonálne prvky sú zase dosť malé. Obecne platí, že kvark s nábojom $+2/3$ (u, c, t) sa transformuje na kvark s nábojom $-1/3$ (d, s, b) a naopak prostredníctvom nabitého $W^{\pm}$ bozónu, ktorý mení náboj o jednotku. Tiež platí, že sa kvarky rozpadajú v postupnosti od najviac hmotných po tie najmenej hmotne 
$$
t\rightarrow b\rightarrow s\rightarrow u \leftrightarrow d
$$
Nasledujúci obrázok graficky znázorňuje prechody medzi kvarkmi \ref{sf1:fig:Kvarky_prechody}
\begin{figure}[!h]
\centering
\includegraphics[width=0.6\textwidth]{Kvarky_prechody.png}
\caption{Diagram znázorňujúci prechodové možnosti medzi kvarkmi prostredníctvom slabej interakcie a indikácie pravdepodobnosti prechodov, ktoré sú dane CKM maticou.}
\caption{}
\label{sf1:fig:Kvarky_prechody}
\end{figure}

\subsection{Silná interakcia}
Silná interakcia je sila pôsobiaca len medzi časticami s nenulovým \textcolor{red}{fa}\textcolor{green}{reb}\textcolor{blue}{ným} nábojom. Tento druh náboju obsahujú iba kvarky a preto táto interakcia nie je univerzálna. Táto interakcia je pôsobí len na malých vzdialenostiach medzi hadrónmi a jej dosah je približne $10^{-15}\,m$. Jej prejavmi sú 
\begin{itemize}
	\item jadrové sily medzi nukleónmi v jadre  
	\item sily, ktoré držia kvarky pohromade v nukleóne
	\item produkcia častíc pri vysokoenergetických zrážkach hadrónov
\end{itemize}
Okrem toho, že silná interakcia nie je univerzálna, čiže platí len pre kvarky, tak je aj obmedzená väčším počtom zákonov zachovania ako ostatne interakcie.

Silnej interakcii prislúcha väzbová konštanta, ktorá charakterizuje jej veľkosť
$$
\alpha_s= \frac{g_s^2}{4\pi},
$$
kde $g_s$ je náboj konštituentného kvarku. Pre malé energie je hodnota tejto konštanty $\alpha\approx 1$. Väzbová konštanta pre silnú interakciu je oveľa väčšia ako pre elektromagnetickú interakciu. Veľkosť tejto konštanty pre malé energie ma za následok nepoužiteľnosť poruchovej teórie kvôli divergentným členom. Avšak táto konštanta ma tu vlastnosť, že jej veľkosť závisí od prenesenej energie (resp. hybnosti), preto sa tato konštanta nazýva aj $bežiaca$ väzbová konštanta. S rastúcou energiou interakcie (s rastúcou hybnosťou zrážajúcich sa častíc) totiž táto väzbová konštanta klesá, čo vedie k asymptotickej voľnosti (divergentne členy začnú konvergovať, čo umožni použiť poruchovú teóriu). Závislosť $\alpha_s$ na hybnosti je nasledujúca
$$
\alpha_s\approx\frac{12\pi}{(11n_c-2n_f)\ln\big(\frac{k^2}{\Lambda^2}\big)}
$$ 
kde $n_c$ je počet farebných nábojov, $n_f$ je počet kvarkových druhov častice (flavour) a $\Lambda$ je škálovací parameter vychádzajúci z renormalizačného procesu a má hodnotu približne $200\,MeV$. (Napr. $\alpha_s=0.12$ pre $k^2=(100\,GeV)^2$.)

Mediátorom silnej interakcie je vektorová častica gluón, ktorá je neutrálna a nehmotná, niečo ako fotón pre QED. Avšak, pre elektromagnetickú interakciu máme len dva typy elektrického náboja: kladný a záporný. V teórii silnej interakcie, ktorá je popísaná kvantovou chromodynamikou (QCD), však existuje 6 druhov náboja, ktorý sa z nevysvetliteľnej príčiny nazýva \textcolor{red}{fa}\textcolor{green}{reb}\textcolor{blue}{ný} náboj a sú to tieto, viď obrázok \ref{sf1:fig:Color_quarks}
\begin{figure}[!h]
\centering
\includegraphics[width=0.6\textwidth]{Color_quarks.png}
\caption{6 druhov kvarkov, horne tri sú farebné náboje red, green, blue a spodné sú ich anti-farebné náboje anti-red, anti-green, anti-blue.}
\label{sf1:fig:Color_quarks}
\end{figure}
\newline
Podľa QCD sú baryóny (častice tvorené z 3 kvarkov) a mezóny (častice tvorené jedným kvarkom a anti-kvarkom) farebne neutrálne. Pre gluóny platí, že sú nositeľmi aj jednej farby aj jednej anti-farby súčasne, kebyže to tak nie tak by potom hadróny nemohli byt viazané vo farebne neutrálnom systéme.  Z toho potom máme celkovo $3^2=9$ možných farebných kombinácii pre gluóny. Avšak, ako vieme, nie je to úplne pravda, že ich je 9. V skutočnosti máme len 8 gluónov a to z toho dôvodu, že bezfarebný singletný stav $\frac{1}{\sqrt{3}}(r\bar{r}+b\bar{b}+q\bar{q})$, nebol zatiaľ experimentálne pozorovaný. Ak by totiž tento singletný stav existoval bolo by možné pozorovať silnú interakciu na väčších vzdialenostiach. Poďme si načrtnúť ako to v takom bezfarebnom systéme vlastne funguje. Majme nasledujúci obrázok \ref{sf1:fig:Prechody}
\begin{figure}[!h]
\centering
\includegraphics[width=0.8\textwidth]{Color_changing.png}
\caption{}
\label{sf1:fig:Prechody}
\end{figure}
\newline











Na 1. obrázku máme systém, ktorý je bezfarebny a zatial neprebieha ziadna vymena gluonu. Na 2. obrazku vsak uz mame gluon, ktory sa uvolnil z modreho kvarku. Kedze tento gluon pochadza z modreho kvarku tak jeho farebna polovica musi byt modra. Ta anti-farebna cast gluonu moze byt prakticky hocijaka. V nasom pripade je anti-zelena. Kedze sa odnasa anti-zelena farba tak kvark musi byt zeleny aby bol cely system stale farebne neutralny. Na 3. obrazku sa gluon obsarboval do zeleneho kvarku. Tam sa spolocne vybili anti-zelena a zelena farba a jedine co z gluonu ostalo je modra farba. Navyse gluony maju tu vlastnost, ze mozu interagovat medzi sebou, to fotony napriklad nemozu. Takze, ked mame system, kde je viacej gluonov, moze dojst k tomu, ze gluony navzjom budu interagovat, co moze viest k tomu, ze sa zmeni ich celkovy farebny naboj. Avsak stale sa nesmie zmenit celkovy farebny naboj systemu. Pre nazornejsie a krajsie vysvetlenie odporucam si pozriet toto video (Introduction to subatomic physics and subatomic particles: Part III na YOUTUBE).\par
Vzhladom k tomu, ze su gluony nehmotne je mozne ocakavat, ze cast statickeho QCD potencialu bude podobna QED potencialu. Tvar QCD potencialu je 
$$
V_{QCD}=-\frac{4}{3}\frac{\alpha_s}{r}+kr.
$$ 
Tento potenacial sa nazyva Cornell-ov potencial. Faktor $4/3$ vyplyva z toho, ze mame 8 farebnych gluonov, ktore moze posobit na 3 kvarky o roznych farbach. Vidime, ze pre male hodnoty $r$ dominuje negativna cast potencialu a nastava \textit{assymtotic freedom} a mozme pouzit one-gluon exchange (to je vlastne len to, ze mozme pocitat poruchovu teoriu s predpokladom, ze sa tam vymiena gluon, podobne nieco ako pre foton ked sa vymiena). Druhy clen je asociovany s viazanostou kvarkou. Pre velke vzdialenosti je potencialna energia medzi kvarky taka velka, ze v istom momente sa tato energia premeni na novo vzniknuty kvark-antikvark par. Takze namiesto toho, aby sme dostali oddeleny kvark a anti-kvark, dostaneme dva pary kvark-antikvark.\par
Vertex faktor silnej interakcie pozostava z kvarkov a gluonov. Zakladny vertex sa sklada z dvoch fermionovych liniek a jednej bozonovej. Ako sme uz spominali vysie, na rozdiel od QED je mozne v tomto pripade mat aj dva vertexy, ktore zahrnuju troj- a stvor- gluonovu interakciu. Toto je zakladny rozdiel oproti QED, kde fotony navzajom medzi sebou neinteraguju. Vyskyt z tychto vertexov v QCD je mozny vdaka ne-Abelovskej kalibracnej transformacii. Je to velmi podobne tomu, co sme mali pre elektroslabu interakciu. Aj tam sa totiz nachadzaju pripady, kedy dochadza ku troj- a stvor- bozonovej interakcii, ktora je taktiez podmienena touto ne-Abelovskou kalibracnou transformaciou. Akurat tam medzi sebou interaguju $W^{\pm}, Z^0$ a $ \gamma$ bozony. Porzi nasledujuci obrazok \ref{sf1:ref:vertexy}.
\begin{figure}[!h]
\centering
\includegraphics[width=0.5\textwidth]{Vertexy.png}
\caption{}
\label{sf1:ref:vertexy}
\end{figure}
\newline
Kedze sme sa uz dostali k tym kalibracnym transformaciam, zadefinujme si, co to ta QCD vlastne je. QCD je typ kvantovej teórie poľa zvanej teória ne-Abelovskej kalibracnej transformacii so skupinou symetrií SU(3) navrhnuta David-om Gross-om, David-om Politzer-om, and Frank-ok Wilczek-ok . Lgrangian pre 
QCD ma tvar
\begin{equation}
\begin{gathered}
\mathcal{L}_{QCD}=-\frac{1}{4}G^a_{\mu\nu}G^{a\mu\nu}+
\sum_{\psi}\bar{\psi}_i\big(i\gamma^{\mu}(\partial_{\mu}\delta_{ij}-ig_sG_{\mu}^aT_{ij}^a)-m_{\psi}\delta_{ij}\big)\psi_j\\
G_{\mu\nu}=\partial^{\mu}A^{\nu}-\partial^{\nu}A^{\mu}+gf_{abc}A^{\mu}_bA_c^{\nu}
\end{gathered}
\end{equation}
kde $G^{\mu\nu}$ je antisymetricky tenzor, ktoreho posledny clen kompenzuje nekomutativnost rotacii vo farebnom priestore, a tento posledny clen moze za tu troj- a stvor- gluonovu self-interakciu, $\psi_i$ je Dirakov spinor kvarkoveho pola s farbou i=(r,q,b), $G_{\mu}^a$ je 8 komponentne SU(3) kalibracne pole, $T_{ij}^a$ reprezentuje 3x3 Gell-Mannovu maticu, $g_s$ silna vazbova konstanta.\par
Existuje vela sposobov ako nalozit s QCD. Da sa k nej pristupovat pomocou poruchovej teorie, ktora je zalozena na assymptotickej volnosti (male $\alpha_s$). Medzi neporuchovýma teoriema ma najsilnejsie postavenie tkz. Lattice QCD - k redukcii analytickych integrabilnych drahovych integralov sa na numericke vypocty poziva sada diskretnych bodov rozlozenych na mriezke (lattice). Pre riesenie specifickych problemov sa pouzivaju efektivne teorie, ktore v istych limitach davaju kvalitativne presne vysledky. Takouto teoriou je napriklad Chiralna poruchova teoria, co je efektivna teoria pre QCD pri nizkych energiach.\par
\textbf{Hadronizacia} alebo tiez fragmentacia je formovanie hadronov z kvarkov a gluonov. Tento jav moze nastat po vysoko-energetickych zrazkach v collideri castic, kde su produkovane "volne" kvarky a gluony. Takato produkcia parov moze vzniknut napriklad anihilaciou pri interakcii $e^-e^+$. Nasledne medzi kvarkom a antikvarkom nastava dynamicka separacia. Ta nastane preto, lebo tieto castice maju taku velku energiu, ze sila, ktora ich drzi po kope nie je dostatocne velka, aby tomu zabranila. Su dva pristupy ako kvantitativne pochopit proces formovania hadronov: 
\begin{itemize}
	\item Chromostatic \newline
	Kvark-antikvark par vytvori dalsi kvark-antikvark par, akonahle vzdialenost medzi prvym povodnym parom je radovo $1\,fm$. Pri takejto vzdialenosti je hustota energia medzi kvarkmy natolko velka, ze dojde k vytvoreniu dalsieho paru. Tento proces pokracuje az kym relativna hybnost kvarkov neklesne na taku hodnotu, ze uz nebude moct dochadzat k tvoreniu dalsich parov, pozri obrazok \ref{sf1:fig:anihilation}. Hadrony nasledne vznikaju pozdlz retazca tvorenia kvarkov. Vytvorene hadrony produkuju sprsku zvanu jety, ktore su priblizne v smere prveho kvarku, anti-kvarku.
	\begin{figure}[!h]
	\centering
	\includegraphics[width=0.3\textwidth]{Anihilation.png}
	\caption{}
	\label{sf1:fig:anihilation}
	\end{figure}
	\item Chromodynamical \newline
	Tento pristup sa odohrava tkz. kvark-gluonovou kaskadou, pozri obrazok \ref{sf1:fig:cascade}. Zacina emisiou gluonu kvarkom alebo anti-kvarkom. Tento gluon moze produkovat bud kvark-antikvark par alebo gluonovy par. Kedze je viacej gluonov ako kvarkov tak statisticky sa tento gluon bude rozpadat viacej do gluonov ako do kvarkov. Silna vazba bude narastat so zmensujucou sa hodnotou hybnosti virtualnej castice. Na konci kaskady kvarky vytvoria bezfarebne viazane stavy. Je jasne, ze tento model nemoze byt pouzity az na koniec hadronizacneho procesu. Dovodom je, ze pre male hodnoty hybnosti sa vazbova konstanta zvacsuje a tym sa narusa poruchovy rozvoj. V tejto oblasti prevláda elasticky efekt, ktorý skončí tvorbou hadronou. Hadrony su tvorene vo vakuu na konci kvark-gluonovej kaskady. Transverzalna hybnost hadronov vzhladom na povodny smer kvarkuje je limitovana Heisenbergovym principom neurcitosti. Hadrony su preto koncentrovane okolo povodneho smeru kvarku a tvoria jety. Ak ma prvy gluon dostatocne velku transverzalnu hybnost, tak moze vzniknut treti hadronovy jet v smere tohto gluonu.
	\begin{figure}[!h]
	\centering
	\includegraphics[width=0.3\textwidth]{Cascade.png}
	\caption{}
	\label{sf1:fig:cascade}
	\end{figure}
\end{itemize}
\par
Dynamika tychto hadronizacnych procesov stale nie je uplne pochopena pomocou QCD a to vdaka tomu, ze poruchova teoria v QCD, formulovana pomocou kvarkov a gluonov, je platna len na malych vzdialenostiach. Na vacsich vzdialenostiach sa tato poruchova teoria zhrouti. Existuju vsak rozne fenomenologicke modely, ktore sa to snazia popisat. Prvym takymto modelom bol Feynmanov a Fieldov, nezavisle vytvoreny, fragmentacny model. Zakladnou myslienkou tohto modelu je predstava hodronizacie kvark-dikvarkoveho systemu ako nezavislej fragmentacie kvarku a dikvarku. Tento predpoklad je ale v principe neudrzatelny, pretoze k hadronizacii dochadza vdaka vzajomnej interakcii medzi nimi. Avsak ukazalo sa, ze vysledne rozdelenie hadronov moze byt v istom priblizeni popisane fragmentacnou funkciou $D^h_q(k,p_T)$ (jazyk fragmentacneho modelu). Ta popisuje pravdepodobnost, ze parton $q$ vytvori hadron $h$ nesuci cast $k$ z povodnehej energie partonu a pricnou hybnostou $p_T$. Tato funkcia ako kazda ina distribucna funkcia by mala byt univerzalna t.j. nezavisla na procese. \par
\textbf{Jet} je sprska castic, ktora sa nachadza v uzkom kuzely, ktora vznika pri hadronizacii kvarkov a gluonov. Su to vlastne experimentalne znaky kvarkov a glunov produkovanych vo vysoko-energetickej fyzike, pozri obrazok \ref{sf1:fig:jet}. 
\begin{figure}[!h]
\centering
\includegraphics[width=0.7\textwidth]{Jet.png}
\caption{}
\label{sf1:fig:jet}
\end{figure}
Skutocnost, ze smery a energie jetov dobre odpovedaju smerom a energiam povodnych kvarkov, nie je triavalna vlastnost procesu hadronizacie. Smerove rozdelenie jetov v priestore vzhladom ku smeru $e^-e^+$ zrazky by malo byt rovnake ako koncovy stav pri procese $e^-e^+\rightarrow \mu^-\mu^+$, pretoze miony a kvarky maju spin 1/2. Tento fakt je jednym zo silnejsich dokazov toho, ze kvarky maju spin 1/2. Kvark, anti-kvark a gluon mozu fragmentovat do hadronov, co vedie ku troj-jetovym eventom. Vzhladom k tomu, ze uhlove rozdelenie jetov je v sulade s teoretickou predpovedou pre gluon so spinom 1, poskytly tieto eventy jednoznacny dokaz o existencii gluonov.
\subsection{Gravitačná interakcia}
Tato sila sa uplatnuje len pri silovom posobeni medzi makroskopickymi objektmy a v kozmickej mechanike. V subatomarnej fyzike nehra podstatnu rolu a preto moze byt zanedbana. Starsia teoria gravitacie pochadza od Newtona, ktory tuto silu popisal ako $$ \vec{F}=\kappa\frac{m_1m_2}{r^2}\frac{\vec{r}}{r}, $$ kde $\kappa=6.672\times10^-11\,m^3kg^-1s^2$, je vazbova konstanta. Jej posobenie je nekonecne, podobne ako pre elektromagneticku interakciu. Da sa povedat, ze to je najdemokratickejsia sila aku pozname. Pretoze je univerzalna a na vsetky telesa posobi rovnako. Modernou teoriu gravitacie je Vseobecna teoria relativity, ktora tvrdi, ze zrychlenie a gravitacna sila je ta ista vec. Sprostredkovatelom tejto interakcie je graviton (zatial nebol pozorovany), ktory by mal byt nehmotny a mal by mat spin=2.
\section{Vlastnosti leptónov a hadrónov}
Sem si pripomenieme vlastnosti, ktore sme hore neuviedli.
A ak spomiem nieco tak to je dolezite a je dobre si to znovu pripomenut.\newline
\textbf{Leptony}
\begin{itemize}

\item Leptony nemaju farebny naboj a tak nepodliehaju silnej interakcii podobne ako neutrina, ktore nemaju elektricky naboj a tak nepodliehaju elektromagnetickej interakcii. Neutrina interaguju jedine slabo.
\item Kedze leptony maju spin, mozu vytvarat magneticke pole. Velkost magnetickeho dipoloveho momentu je dany 
$$\mu=g\frac{Q\hbar}{4m},$$ kde $m$ je hmotnost leptonu, $g$ je tkz. g-faktor pre lepton. Prvy rad pribliznej kvantovej mechaniky predpoveda tuto hodnotu rovnu 2 pre vsetky leptony. Avsak, vyzsie rady kvantoveho efektu sposobuju korekciu tejto hodnoty oznacovanu ako anomalny magneticky moment. Toto cislo je velmi citlive na vela detailov a preto jeho spocitanie a nasledne experimentalne zmeranie bolo obravskym uspechodm QED. Vo Feynmanovych diagramoch toto cislo reprezentuju slucky. Tato korekcia ma priblizne hodnotu $a_e=0.001159....$ 
\item vsetky leptony, okrem tau neutrina, boli pozorovane priamo v experimentoch - ako volne castice.
\item Presne merania mionovych vlastnosti boli vykonane prostrednictvom skumania mezoatomov vytvorenych zachytenim mionu v atomu na Bohrove orbite. Hodnota energie stavu mezoatomu s hlavnym kvantovym cislom $n$ je linearne umerna hmotnosti mionu$$E(n)=-\frac{Z^2e^4m}{2(4\pi\epsilon_0)^2\hbar^2n^2}$$ a teda je v absolutnej hodnote asi 200-krat vacsia nez energie odpovedajuce tomu istemu stavu ale s elektronom. Prechod do stavu z nizsou hladinou energie sposobuje emisiu rentgenoveho ziarenia, ktore je charakteristicke a mozno z neho vyvodit informacie o naboji, hmotnosti a spine.
\item Doba zivota mionov je okolo $\tau_0=2.2\mu s$ a ich rychlost je $\beta=0.98$. Velka cast mionov vznika vysoko v atmosfere, asi 10km nad zemou. Kedze sa pohybuju tak rychlo nastava dilatacia casu vzhladom na pozorovatela na Zemi. To znamena, ze vo svojej sustame ma mion dobu zivota tych $\tau_0$ ale v sustave pozorovatela, ktory stoji na zemskom povrchu a sleduje miony je ten cas o cosi dlhsi presnejsie $\tau=\tau_0/\sqrt{1-\beta^2}\approx11\mu s$. Takze drahu, ktoru prejde mion urcime velmi jednoducho  $l=\beta c\tau \approx 13.2\,km$. A preto je mozne, ze pozorujeme kozmicke miony na povrchu Zeme. Toto je aj jeden z dokazov Einsteinovej teorie relativity.
\item dominantny rozpad mionu: $\mu^-=e^-+\bar{\nu}_{e^-}+\nu_{{\mu}^-}$. Ostatne mozne rozpady např. $\mu^-=e^-+\gamma$ je sice kinematicky mozny ale nezachovava sa flavor number. Taketo rozpady maju $BR\sim10^{-12}$.
\item Helicita částice je vyjadrenie orientacie medzi spinom castice a jej hybnostou. Castice, ktorych spin je orientovany v rovnakom smere ako hybnost, su pravo-tocive a v pripade, ze orientacia je v protismere tak hovorime o lavo-tocivych casticiach. Pre QCD a QED su castice rovnako pravo a lavo tocive, nedochadza tam k ziadnej asymetrii. Avsak, v pripade slabej interakcie mame len lavo-tocive fermiony a pravo-tocive neutrina-maximalne narusenie parity.
\item Elektricky naboj moze byt spocitany z projekcie spinu a slabeho-hypernaboja cez $Gell-Mann-Nishijima$ formulu $$Q=T_3+\frac{Y_W}{2}$$.
\item Kineticka energia $\beta$ castica ma spojite spektrum od 0 az po maximalnu predanu energiu. Typicka energia je $1\,MeV$, ale extremnych pripadoch to moze byt aj niekolko $10\,MeV$. Fundamentalny $\beta$ rozpad nastava vdaka konverzii d-kvarku nautronu na u-kvark protonu emisiou $W^-$ bozonu, ktory sa nasledne rozpada na $e^-$ a $\nu_{e^-}$.
\item $\Gamma\sim KG_F^2m_l^5$, kde $K$ je ciselna konstanta, $G_F$ je Fermiho konstanta a $m_l$ je hmostnost leptonu. Stredna doba zivota je $\tau=\hbar/\Gamma$
\item Kvarky, ktore urcuju vlastnosti hadronov sa nazyvaju valencne kvarky. Hadrony naviac obsahuju tiez prchave kvark-antikvark pary, ktore nemenia vlastnosti, ale prispievaju k jeho kludovej energii. Tieto kvarky sa nazyvaju morske kvarky.
\item $$ R=\frac{\sigma(e^+e^-\rightarrow q\bar{q})}{e^+e^-\rightarrow \mu^+ \mu^-}=3\Sigma_{q}e^2_q$$ tento pomer je zalozeny na tom, ze existuju tri farebne stavy kvarkov a je na energii takmer nezavisly. Porovnanim experimentach dat vychadza, ze to sedi.\newline
\end{itemize}
\textbf{Hadrony}
\begin{itemize}
	\item Vlastnosti hadronov ako naboj, spin, atd. su urcene valencnymi kvarkmy, zatialco hmotnost hadronov ma s valencnymi kvarkmy velmi malo spolocne a velka cast hmotnosti pochadza z mnozstva energie, ktoru prenasaju gluony.
	\item Hadrony sa rychlo rozpadaju silnou interakciou, pokial im to umoznia kvantove cisla (zakony zachovania kvantovych cisiel). Dalej potom uz pracuje slaba interakcia, ktora meni vonu kvarkov az na kvarky prvej generacie a nakoniec az na nejake leptony.
	\item Najtazsie zname castice, vznikajucí pri casticovych interakciach pri vysokych energiach, su hadrony zvane hyperony. Vsetky hyperony vykazuju silnu interakciu a su vysoko nestabilne s velmi kratkou dobou zivota. Vzhladom k tomu, ze hyperony interaguju silno, mozu vstupovat do jadier a byt tam naviazane jadrovymi silami - vzniknu hyperjadra. V typickom hyperjadre je jeden nukleon nahradeni hyperonom. Su to nestabilne utvary, ktore sa rozpadaju dvojako: bud mezonovym rozpadom alebo nukleonovym rozpadom.
	\item Hypernaboj je definovany ako $Y=B+S+C+T+\tilde{B}$, kde jednotlive znaky su baryonove cislo, podivnost, povab, topness a beauty - kvantove cisla. Nasledne vdaka tomu mozme spocitat priemet izospinu $I_3=Q-Y/2$, kde $Q$ je elektricky naboj.
\end{itemize}
\section{Symetrie a zákony zachovania}
\textbf{Izospin}\par
Fyzikální veličina, kterou zavedl v roce 1935 Eugene Paul Wigner, aby mohl popsat multiplety různých částic. Jedná se o kvantové číslo související se silnou interakcí. Částice, na něž působí silná interakce shodně, ale které mají různý elektrický náboj, lze považovat za jedinou částici s hodnotou izospinu související s počtem nabitých stavů. Izospin je bezrozměrná veličina a její název je odvozen od skutečnosti, že matematické struktury, které popisuje, jsou podobné těm, které popisuje vnitřní moment hybnosti, zvaný spin.\par
Některé částice mají mnoho společných znaků, proto je možné chápat je jako obměny jediného objektu. K rozlišení těchto stavů se zavádějí různá kvantová čísla, z nichž nejčastější je izospin. Takové skupiny příbuzných elementárních částic se nazývají multiplety. Částice v multipletu se vzájemně liší projekcí izospinu. Všechny částice multipletu mají stejnou velikost izospinu a liší se její projekcí do libovolné osy, pozri obrazok \ref{sf1:fig:Baryon_decuplet}.\par
\begin{figure}[!h]
\centering
\includegraphics[width=0.5\textwidth]{Baryon_decuplet.png}
\caption{Kombinácia troch u, d alebo s-kvarkov tvoriacich baryóny so spinom=3/2 tvorí baryónový decuplet.}
\label{sf1:fig:Baryon_decuplet}
\end{figure}
Částice v rámci jednoho multipletu se od sebe odlišují převážně elektrickým nábojem, popsaným zetovou složkou izospinu. Naopak příbuzné částice v multipletu mají stejnou hodnotu spinu, baryonového čísla a podobnou klidovou hmotnost.\par
Zjistilo se, že při procesech způsobovaných silnou interakcí se hodnota izospinu zachovává, zatímco v procesech elektromagnetické interakce se hodnota izospinu může zvýšit nebo snížit o jedničku.\par
Počet částic v multipletu je dán hodnotou izospinu. Kupříkladu pro nukleon je hodnota izospinu 1/2, multiplet má tedy dvě částice neutron a proton a nazýváme ho dublet. Je-li hodnota izospinu 1, má multiplet 3 částice, příkladem může být kladný, záporný a neutrální pion, takový multiplet nazýváme triplet. Existují též singlety, pro ně je izospin roven 0. Příbuzné částice v multipletu, například proton a neutron, lze považovat za různé kvantové stavy jediné částice = nukleon. Izospin tyto částice odlišuje.\newline
\textbf{Dalsie hadronove cisla}\par
Vedla izospinu je tu este sada kvantovych cisiel, ktore su charakteristicke len pre hadrony. Su to baryonove cislo B, podivnost S, povab C, krasa B, topness T. 


Obrazok \ref{sf1:fig:zzachovania} znazornuje veliciny a ich zachovavanie sa v roznych interakciach. 
\begin{figure}[!h]
\centering
\includegraphics[width=0.9\textwidth]{Zakony_zachovania.png}
\caption{Tu len pripomeniem, ze baryonove cislo je definovane ako $B=\frac{1}{3}(n_q+n_{\bar{q}})$. Pre leptonove cislo plati $L=n_l-n_{\bar{l}}$, tu musi byt zachovany aj flavor leptonu. Napriklad takyto proces nie je pozorovany: $\mu^- \rightarrow e^-+\gamma$.}
\label{sf1:fig:zzachovania}
\end{figure}
\newline
Skor ako prejdeme ku samotnym zakonom vysvetlime najprv vyznam zakonu zachovania. So zakonmy zachovania velmi suvisia transformacie. Predpokladajme, ze mame system popisany lubovolnymi suradnicami, napr. $\vec{r}=x,y,z.$ Nasledne posunieme system po osi $x$ o vzdialenost $a$. Prepokladajme, ze fyzikalny popis systemu sa tymto nezmeni, tzn. chovanie systemu je invariantne voci pousnutiu pozdlz osi $x$.\par
V teoretickej fyzike existuje teorem, ktory spojuje invarianciu vzhladom k danej transformacii so zachovavajucou sa velicinou-\textbf{Noetherovej teorem}: Kazdej grupe transformacii suradnic zavislych spojito na realnom parametri, pri ktorych Lagrangeova funkcia zostava invariantna, odpoveda prvy integral Lagrangeovych rovnic tejto sustavy $=$ zakon zachovania. V nasom pripade invariancia vzhladom k posunutiu v $x$-ovej osi sa teda dostaneme k zachovaniu $x$-ovej zlozky hybnosti. Tato invariancia sa nazyva symetria systemu.\par 
Uvedieme zhrnutie, obrazok \ref{sf1:fig:zzachovania}, tychto spojitych transformacii a k nim pridruzime zachovavjuce sa veliciny za predpokladu, ze system je invariantny voci danym transformaciam
\begin{figure}[!h]
\centering
\includegraphics[width=0.5\textwidth]{spojite_trans.png}
\caption{Spojite transformacie a s nimi spojene zachovavajuce sa veliciny. Vsetky spojite transformacaie su spojene s aditivnymi kvantovymi cislamy, aditivne v tom zmysle, ze vsetky prispevky roznych casti systemu sa scitaju do celkovej hodnoty.}
\label{sf1:fig:spojtran}
\end{figure}
\newline
Spojite transformacie maju tu vlastnost, ze kazda transformacia moze byt vyjadrena ako sucet malych transformacii. Opakom k tymto transformaciam su transformacie diskretne, ktore nemoze byt vyjadrene pomocou mensich transformacii. Medzi diskretne veliciny patri Parita, Nabojove zdruzenie alebo Time reversal.\newline

\textbf{Parita} \par
Pred rokom 1956 fyzici verili, ze zrkadlovy obraz akehokolvek fyzikalneho procesu reprezentuje dalsi mozny fyzikalny proces. A prave v tomto roku bol Lee-om a Yang-om navrhnuty experimentalny test, ktory mal zistit ci je to pravda, a to aj pri posobeni slabej interakcie. V tomto experimentu boli poctivo zrovnane spiny jadier $^{60}Co$ tak, aby mierili vsetky do jedneho smeru (povedzme, ze napriklad hore). Kobalt sa nasledne rozpadol beta rozpadom a bol pozorovany smer vyletujucich elektronov. Tento smer pre podstatnu vacsinu elektronov bol v smere spinu jadier kobaltu.\par 
Toto jednoduche pozorovanie malo vsak udivujuce nasledky. Predpokladajme, ze pozorujeme zrkadlovy obraz tohto procesu. Obraz jadra rotuje opacnym smerom (spin smeruje dolu). Zrkadlove elektrony ale aj tak vylietavaju smerom hore, ako to bolo v predchadzajucom pripade. V zrkadlovom odraze su tak elektrony emitovane v smere opacnom k smeru spinu jadier. Mame tak fyzikalny proces, ktoreho zrkadlovy obraz nepozorujeme v prirode. Parita sa tak pri slabych interakciach nezachovava (pokial by sa zachovavala tak elektrony by boli emitovane rovnomerne v oboch smeroch). Nezachovavanie parity je stopou slabej interakcie.\par
Najviac zretelne je narusenie paity v spravani neutrin. Vieme, ze neutrina su lavotocive a anti-neutrina su pravotocive. Relativne jednoduchou nepriamou metodou merania helicity neutrin je vyuzitie rozpadu pionu: $\pi^- \rightarrow \mu^-+\bar{\nu}_{\mu}$. Pokial je pion v klude, mion a anti-neutrino su emitovane v opacnom smere. Kedze pion ma nulovy spin, spiny mionu a anti-nutrina musia byt opacne. V pripade, ze anti-nutrino je pravotocive, musi byt aj mion pravotocivy (v kludovej sustave pionu), co je overene experimentalne (to ze su oba pravotocive vychadza z definicie helicity-smer spinu je paralelny so smerom hybnosti danej castice).\par
Navzdory narusenia parity v slabej interackii, v silnych a elektromagnetickych interakciach sa parita zachovava. Je tada uzitocne vytvorit formalizmus a terminologiu pre operaciu parity. Oznacme operator parity ako $\hat{P}$. Pokial tento operator aplikujeme na vektor $\vec{a}$, tak vytvorime vektor do opacneho smeru: $\hat{P}\vec{a}=-\vec{a}$. Uvazujeme teraz vektorovy sucin $\vec{c}=\vec{a}\times \vec{b}$. Operator parity zmeni znamienko obidvom vektorom, a tak vektorovy sucin nazmeni znamienko: $\hat{P}\vec{c}=\vec{c}$. Podobna situacia je aj pre skalary. Operacia parity moze byt zhrnuta nasledovne 
\begin{equation*}
\begin{gathered}
Skalar: \hat{P}s=s  \hspace{2cm} Pseudoskalar: \hat{P}p=-p    \\
Vektor: \hat{P}\vec{v}=-\vec{v}  \hspace{2cm} Pseudovektor: \hat{P}\vec{a}=\vec{a}
\end{gathered}
\end{equation*}
Pri dvojnasobnej aplikacii operatora parity dostaneme povodny stav, plati teda $\hat{P}^2=I$, $I$ je jednotkova matica. Vlastnemi hodnotami tohoto operatora su $\pm1$.\par
Majme teraz vlnovu funkciu $\psi(\vec{r})$, ktora popisuje urcity system. Ked na nu aplikujeme operator parity dostavame $$ \hat{P}\psi(\vec{r})=\psi(-\vec{r}).$$ Pokial je tato funkcia vlastnou hodnotou tohto operatora, tak podobne ako pre normalny vektor mozeme pisat
$$ \hat{P}\psi(\vec{r})=\pm \psi(\vec{r}),$$ kde vlastne funkcie opetatora $\hat{P}$ s vlastnou hodnotou $(+1)$ nazveme parne (sude), zatialco tie s vlastnou hodnotou $(-1)$ nazveme neparne (liche). V pripade centralnych interakcii, kedy vlnovu funkciu zavislu na $\vec{r}$ mozme napisat ako sucin radialnej vlnovej funkcie a sferickej vlnovej funkcie zavislej na orbitalnom momente hybnosti $l$ a jeho projekcii $m$ do z-tovej osy $$ \psi(\vec{r})= R(r)Y_l^m(\theta, \varphi),$$ mozeme transformovat $\vec{r}\rightarrow -\vec{r},\theta\rightarrow \pi - \theta,\varphi\rightarrow \pi+\varphi$. Odtialto vidime, ze stavy castice pohybujuce sa v poli centralnych sil s parnym orbitalnym momentom hybnosti $l$ maju parnu paritu a stavy s neparnym $l$ maju neparnu paritu.\par
Mimo parity spojenej s orbitalnym pohybom castice zavadziame aj tkz. $vnutornu$ $paritu,$ ktora je bud kladna alebo zaporna. Velmi lahko sa da pochopit v pripade hadronov, ktore maju vnutornu strukturu. Avsak, aj elementarne castice maju vnutornu paritu, ktora je chapana ako charakteristicky rys danej castice.\par
Hadrony su vlastne stavy $\hat{P}$ a je ich mozene klasifikovat pomocou vlastnej hodnoty parity, rovnako ako su klasifikacie pomocou spinu, naboja, izospinu, podivnosti atd. Parita fermionov musi byt opacna k parite odpovedajucej anticastice, parita bozonu musi byt musi byt totazna s paritou danej anticastice. Pokial priradime kvarkom kladnu vnutornu paritu, anti-kvarky ju musia mat zapornu. Parita zlozeneho systemu v zakladnom stave je produktom (sucinom) parit jeho konstituentov (multiplikativne kvantove cislo) - preto maju baryony kladnu paritu a mezony zapornu. Pre excitovane stavy plati $(-1)^l$, kde $l$ je moment hybnosti.\par
Majme napriklad rozpad $\rho_0 \rightarrow \pi^+ \pi^-$. Vieme, ze spin pre $\rho$ je rovny 1 a piony maju spin rovny 0, preto vysledny pionovy stav ma $l=1$. Dalej vieme, ze vnutorna parita $\rho$ mezonu a pionov je (-1). Plati potom $(-1)=(-1)(-1)(-1)^l$, kde v nasom pripade $l=1$. Vidime, ze celkom parita sa zachovava a nic nebrani aby tento proces nastal.\par 
Zoberme si ale teraz pripad $\rho_0 \rightarrow \pi^0 \pi^0$. Tu plati skoro vsetko, co pre predchadzajuci pripad. Nastava tu vsak jedne problem. A to taky, ze mame dva rovnake bozony, ktore sa musia riadot Bose-Einsteinovou statistikou a tak musia vytvorit symetricku funkciu. Avsak pre $l=1$ mame iba anti-symetricku vlnovu funkciu a z toho dovodu tento proces nemoze nastat.
 \newline

\textbf{Nabojove zdruzenie}\par
Vo fyzike elementarnych castic zavadziame operaciu, ktory nazyvame $nabojove$ $zdruzenie$ $\hat{C}$.
Tato operacia konvertuje casticu na jej anti-castice: $\hat{C}\lvert{p}\rangle=\lvert{\bar{p}}\rangle$.\par
Nazov nabojove zdruzenie je vsak trochu nevhodny pretoze $\hat{C}$ mozeme aplikovat aj na neutralne castice a vysledkom prevratena hodnota znamienok u vsetkych vnutornych kvantovych cisiel, tj. naboj, baryonove cislo, leptonove cislo, podivnost atd., pricom hmota, energia, hybnost a spin danej castice zostanu nedoknute. Ronovako ako u parity aj v tomto pripade, ked zaposobime na stav dvakrat dostavame povodny stav, tj. $\hat{C}=I$ a vlastnymi hodnotami su tiez $\pm1$. Pre $\lvert p\rangle$, ktore je vlastnym stavom $\hat{C}$, plati $\hat{C}\lvert p\rangle = \pm\vert p \rangle=\lvert\bar{p} \rangle$, kde $\lvert\bar{p} \rangle$ a $\lvert p \rangle$ sa lisia len znamienkom, co znamena, ze reprezentuju ten isty fyziklany jav. Odtialto je zrejme, ze len tie castice, ktore su svojimi vlastnymi anti-casticami, mozu byt vlastnymi stavmy $\hat{C}$, cize su to fotony a mezony leziace uprostred diagramov Eightfold way.\par
Kedze je foton kvantom elektromagnetickeho pola, ktore meni znamienko pri nabojovom zdruzeni, dava smysel, ze vlastna hodnota nabojoveho zdruzenia fotonu je $-1$. System zahrnujuci castice so spinom $1/2$ a ich anti-casticame v konfiguracii s momentom hybnosti $l$ a celkovym spinom $s$ predstavuju vlastny stav $\hat{C}$ s vlastnou hodnotou $(-1)^{l+s}$. Podla kvarkoveho modelu tak pre mezony plati: pseudoskalary maju $l=0$ a $s=0$, a teda $C=+1$, vektory $l=0$ a $s=1$ maju $C=-1$. $C$ je multiplikativne kvantove cislo a rovnoka ako parita sa zachovava v silnych a elektromagnetickych interakciach. Preto napriklad $\pi^0 \rightarrow \gamma + \gamma$, kde $C=+1$ pred i po reakcii ale nemoze sa rozpadat na tri fotony (pre system pre n fotonov je $C=(-1)^n$). Na druhu stranu $C$ sa nezachovava pri slabych interakciach. Pokial by sme $C$ aplikovali na lavotocive neutrino, dostali by sme lavotocive anti-neutrino, ktore vsak neexistuje. Nabojova verzia akehokolvek procesu s neutrinami preto nie je z fyzikalneho hladiska. \newline

\textbf{Time reversal = Časová inverze}\par
Zmena toku casu $t\rightarrow -t$. Prevratenie toku casu tiez prevrati casovu derivaciu priestorovych velicin, co znamena obratenie vsetkych hybnosti $\vec{p} \rightarrow -\vec{p}$ a momentu hybnosti. $\vec{L} \rightarrow -\vec{L}$. Invariancia vzhladom k tejto transformacii ma za nasledok, ze pokial by sme mali dva procesy, z nich druhy by bol procase opacny k tomu prvemu, boli by oba dva rovnako pravdepodobne. Vdaka tejto invarianci tak mozme pouzit ucinne prierezy atomovych alebo jadrovych reakcii k ucinnym prierezom opacnych reakcii. Zatial nebol najdeny ziaden dokaz toho, ze to tak nie je. 
\begin{figure}[!h]
\centering
\includegraphics[width=0.8\textwidth]{Diskretnetran.png}
\caption{Diskretne transformacie a ich vlasatnosti.}
\label{sf1:fig:Diskretnetran}
\end{figure}
\newline
\textbf{CP symetria a jej narusenie}\par
Naznacili sme, ze slabe interakcie narusuju paritu (P) a aj nabojove zdruzenie (C). Povodne sa komunita fyzikov domnievala, ze kombinacia CP symetrie zostava slabymi interakciami nenarusena. Platilo by totiz $\hat{C}\hat{P}\nu_{L}=\hat{C}\nu_P=\nu_{\bar{P}}$.\par
V 50. rokoch vsak bolo narusenie CP symetrie navrhnuto ako reakcia na objavenie narusenia parity. 
K experimentálnemu potvrdeniu se dospelo v roku 1964 v BNL objavenim anomalii v rozpade neutralneho kaonu. Bolo totiz zistene, ze neutrlane kaony sa mozu premenit na svoje anti-castice a naopak, ale k tymto prechodom nedochadza s presne rovnakou pravdepodobnostou v oboch smeroch = mierne narusenie CP symetrie.\par
V roku 1968 prisiel A. Sacharov s myslienkou, ze by narusenie CP symetrie v silnej interakcii mohlo mat pri vzniku Vesmiru za nasledok prevladanie hmoty nad anti-hmotou. V obdobi pred Velkym zjednotenim interakcii castice X a Y prechody sposobovali nerovnosti medzi kvarkmy a leptonmy. Vdaka naruseniu CP invariancie v silnej interakcii prebiehali tieto procesy mierne nesymetricky a viedli  k velmi malemu poruseniu rovnovahy medzi hmotou a anti-hmotou. Zhruba na jednu miliardu reakcii oboma smermi prebehlo o jednu reakciu viac smerom k hmote. Ked sa Vesmir dostatocne ochladil. doslo k anihilacii latky s anti-latkou. Pri tejto anihilacii vsak na kazdu miliardu castic a anti-castic zostala kvoli naruseniu CP symetrie jedna castica hmoty. Prave z tychto castic je dnesny vesmir postaveny.\par
Narusenie CP symetrie bolo pozorovane az v roku 2004 na detektore BABAR na Stanforde. Pri zrazkach tu vznikali kvarky a anti-kvarky $b$. Sledovane boli rozpady castice $B^0$ a jej anticastice $\bar{B}^0$. Rozpad oboch castic ma možnost prebiehat vela moznostami, z nich bolo tiez mozne sledovat vzacny rozpad na dvojicu pion a kaon $B^0 \rightarrow K^+ \pi^-$ alebo $\bar{B}^0 \rightarrow K^- \pi^+$.\par
V pripade rovnakych vlastnosti hmoty a anti-hmoty by obe reakcie mali prebiehat rovnako pravdepodobne a mali by sa objavovat rovnake pocty kvarkov $K^- \pi^+$ a $K^+ \pi^-$. Skutocnost ale bola ina. V experimente bolo detekovano 910 parov $K^+ \pi^-$ a len  695 $K^- \pi^+$. Sposob rozpadu hmoty a anti-hmoty tak prebieha odlisne.\par
Narusenie CP symetrie je v Standartnom modely zahrnute zavedenim komplexnej fazy v CKM matici popisujucej miesanie kvarkov. V tomto schemate je pre komplexnu fazu, a tudiz narusenie CP symetrie, nevyhnutnou podmienkou existencia najmenej troch generacii kvarkov. Podla CPT teoremu odpoveda narusenie CP symetrie naruseniu invariancie vzhladom ku zmene toku casu. Keby CP bola skutocnou symetriou , potom by prirodne zakony platili rovnako ako pre hmotu tak aj pre anti-hmotu.
\newline 
\textbf{CPT teorem}\par
Vo fyziklanych javoch sa zachovava CPT symetria. Kombinacia vsetkych diskretnych transformacii sa poklada za nenarusenu vo vsetkych fundamentalnych interakciach a zaroven za zakladnu vlastnost fyzikalnych zakonov. CPT teoria konkretne prehlasuje, ze vsetky lokalne interagujuce polia, ktorych Lagrangiany su invariantne vo vlastnej Lorentzovskej transforamacii, su invariantne voci kombinovanej transformacii nabojoveho zdruzenia, parity a casovej inverzie. Experimentalne proverovanie tejto invariancie sa robilo porovnavanim vlastnosti castic s ich anti-casticami. Pokial je totiz CPT teorem spravny, kazda castica musi mat presne rovnaku hmotu a dobu zivota ako jej odpovedajuca anti-castica. Prebehlo mnoho merani parov castica-anticastica, najcitlivejsie overene rozdiely poskytol par $K^0-\bar{K}^0$, pozri obrazok \ref{sf1:fig:MeranieCPT}.
\begin{figure}[!h]
\centering
\includegraphics[width=0.3\textwidth]{MeranieCPT.png}
\caption{Relativne hmotnostne rozdiely medzi casticami a anti-casticami.}
\label{sf1:fig:MeranieCPT}
\end{figure} \newline
Relativne rozdiely medzi hmotnostami castic a anti-castic sa robilo pomocou $$\delta(m)=\frac{m-\bar{m}}{m+\bar{m}} $$.
Zo strednych dob zivota mionov bola stanovena horna medza pomerov na $$ \frac{\tau(\mu^+)-\tau(\mu^-)}{\tau(\mu^+)+\tau(\mu^-)}<10^{-4} $$. 
S velkou presnostou boli zmerane magneticke momenty elektronov-pozitron a mion-antimion. Vysledky su obvykle prezentovane v pojmoch gyromagnetickych faktorov $g$, ktorymi su vyjadrene magneticke momenty castic. Pre elektrony mame $$ \frac{g(e^+)-g(e^-)}{g(e+)+g(e^-)}<10^{-12} $$.
Obdobna velicina pre miony ma hornu medzu $10^{-8}$. Vsetky experimentalne dokazy podporuju invarianciu vsetkych interakcii voci transformacii CPT. Pokial je totiz invariancia niektorej z operacii narusena, musi byt kompenzovana ostatnymi transformaciami. Napriklad narusenie invariancie vzhladom k casovej inverzii, musi byt tiez narusena invariancia vzhladom k CP.\par
Dosledkom CPT symetrie je to, že se zrkadlový obraz našeho vesmíru, teda otočenie všetkych objektov s ich poziciami v lubovolnej rovině (odpovedajuce inverzi parity), obratenie všetkych hybností (odpovedajuce časovej inverzi) a nahradenie vsetkej hmoty antihmotou (co odpovedá nábojovej inverzi), bude vyvíjat presne podla známých fyzikálních zákonov. CPT transformacia zmení náš vesmír na jeho zrcadlový obraz a naopak. \par
Dosledkov platnosti CPT symetrie je hned niekolko
\begin{itemize}
	\item Castice s celociselnym spinom podliehaju Bose-Einsteinovej statistike, zatial co polociselne Fermi-Diracove statistike.
	\item Castice a ich anti-castice maju totozne hmotnosti a doby zivota.
	\item vsetky vnutorne cisla castic su opacne k vnutornym kvantovym cislam prisluchajucich anti-castic.
\end{itemize}

\section{Súradnicové sústavy v subjadrovej fyzike}
Transformace kinematických veličin mezi soustavami, Mandelstamovy proměnné,\newline
\textbf{Mandelstamovy premenne}\par
Mandelstamové premenné sú číselné veličiny, ktoré kódujú energiu, hybnosť a uhly častíc v rozptylovom procese Larentzovo-invariantným spôsobom. Používajú sa na rozptylové procesy dvoch častíc na dve častice, pozri obrazok \ref{sf1:fig:Mandelstam}. V Minkovskeho metrike diag(1,-1,-1,-1) maju tieto premenne nasledujuci tvar
\begin{equation}
\begin{gathered}
s=(p_1+p_2)^2=(p_3+p_4)^2 \\
t=(p_1-p_3)^2=(p_4-p_2)^2 \\
u=(p_1-p_4)^2=(p_3-p_2)^2
\end{gathered}
\end{equation}
\begin{figure}[!h]
\centering
\includegraphics[width=0.2\textwidth]{Mandelstam.png}
\caption{V tomto diagrame castice s $p_1$ a $p_2$ prichadzaju a interaguju zatial co $p_3$ a $p_4$ odchadzaju z interakcie.}
\label{sf1:fig:Mandelstam}
\end{figure}
Kazdej z tychto premennych odpoveda urcita topologia zrazky. Tieto typy odpovedaju roznym Feynmanovym diagramom. s-kanal, t-kanal, u-kanal, pozri obrazok \ref{sf1:fig:Kanaly}.
\begin{figure}[!h]
\centering
\includegraphics[width=0.6\textwidth]{Kanaly.png}
\caption{s, t, u -kanaly. s-kanál je jediný spôsob, akým môžu byť objavené rezonancie a nové nestabilné častice za predpokladu, že ich životnosť je dostatočne dlhá a že sú priamo zistiteľné. t-kanál predstavuje proces, v ktorom častica 1 emituje intermedialnu časticu a stáva sa konečnou časticou 3, zatiaľ čo častica 2 absorbuje intermedialnu časticu a stáva sa 4. Pre u-kanal zamenime v t-kanaly len 3 a 4. 
}
\label{sf1:fig:Kanaly}
\end{figure}
\newline
V relativistickej limite zanedbame hmotnosti oproti hybnosti $p^2 >>> (m_0c^2)^2$.\par
Suma Mandelstam-ovych premennych nam da sumu hmotnosti castic. $$ s+t+u=m_1^2+m_2^2+m_3^2+m_4^2 $$\newline
\textbf{Lorentzovske transformacie}\par
Uvazujme dve kartezske vztazne sustavy $S$ a $S^{\prime}$, tak ze ich pociatky splivaju v case $t=t^{\prime}=0$. Suradnicove osy oboch sustav su vzajomne rovnobezne a pohybuju sa tak, ze sustava $S$ (kludova) zostava v pokoji a sustava $S^{\prime}$ sa vzhladom na $S$ hybe rychlostou $v$ v kladnom smere osy $x$. Lorentzove transformacie potom su 
\begin{equation} 
x^{\prime}=\frac{x-vt}{\sqrt{1-\frac{v^2}{c^2}}},\hspace{0.3cm} y^{\prime}=y,\hspace{0.3cm} z^{\prime}=z,\hspace{0.3cm} t^{\prime}=\frac{t-\frac{vx}{c^2}}{\sqrt{1-\frac{v^2}{c^2}}} 
\end{equation}
kde $\beta=v/c$ a $c$ je ryclost svetla vo vakuum. Pre transformacie energie a hybnosti z jednej do druhej sustavy dostavame 
\begin{equation}
E^{\prime]}=\frac{E-p_xv}{\sqrt{1-\frac{v^2}{c^2}}}, \hspace{0.8cm} p_x^{\prime}=\frac{p_x-\frac{vE}{c^2}}{\sqrt{1-\frac{v^2}{c^2}}}
\end{equation}\newline
\textbf{Kinematika zrazkovych procesov}\par
Experimentalne zrazkove procesy: experiment s pevnym tercom, experiment s naproti iducimi zvazkamy, pozri obrazok \ref{sf1:fig:Zrazky}.
\begin{figure}[!h]
\centering
\includegraphics[width=0.8\textwidth]{Zrazky.png}
\caption{Rozne druhy zrazkovych procesov. Vlavo Lab frame a vpravo CMS frame.}
\label{sf1:fig:Zrazky}
\end{figure}
Sustavy su pred aj po interakcii izolovane. Oblast, kde sa castice stretnu a interaguju sa nazyva $interakcna$ $oblast$. Mimo tuto oblast sa castice pohybuju volne (realne je sila interakcie zanedbatelne mensia ako pre oblast interakcie). Budeme pouzivat znacenie velicin nasledovne: pred interakciou budu veliciny bez ciarky a po zrazke s ciarkou.\par
Zakony zachovania energie a celkovej hybnosti izolovanej sustavy mozme napisat v tvare: $E=E^{\prime}$ a $\vec{P}=\vec{P}^{\prime}$. Podla produktov rozlisujeme nasledujuce dva druhy interakcii
\begin{itemize}
	\item \textbf{pruzny rozptyl}: nemenia sa kludove hmotnosti ani typy castic po interakcii. Zo zakona zachovania energie plynie, ze celkova kineticka energia sa zachovava.
	\item \textbf{nepruzny rozptyl}: pri interakcii sa menia hmotnosti zucastnenych castic. Velicinu $Q=[(m_1+m_2)^2-(m_1^{\prime}+m_2^{\prime})^2]=(M-M^{\prime})^2$ nazyvame energia interakcie. Zo zakona zachovania plynie $T^{\prime}=T+Q$. Pre nepruznu interakciu je $Q$ nerovne nule a naopak pre pruznu zrazku je to $Q$ nulove. 
\end{itemize}
\textbf{Suradnicove systemy}\par
\begin{itemize}
	\item \textbf{Laboratorna sustava} - tato sustava je pevne spojena s detektorom. Jej pouzitie nie je vzdy vhodne, kedze vztahy popisujuce interakciu su v nej dost zlozite. V tejto sustave sa meraju hlavne experimenty s pevnym tercom. Avsak nemusi to byt vzdy sustava spojena s detektorom. Pouziva sa aj laboratorna sustava spojena s nejakou casticou, s ktorou ma druha castica interagovat.
	\item \textbf{Taziskova sustava (CMS)} - tato sustava je v pokoji. Zaujima nas len relativny pohyb castic. Celkova hybnost castic je rovna nule, co dost moze zjednodusit vypocet.
	\item \textbf{Tercikova sustava} - sustava v ktorej je hybnost terca nulova.
	\item \textbf{Sustava zvazku} - sustava, v ktorej je hybnost zvazku nulova.
	\item \textbf{Coliding beam frame} - sustava, v ktorej sa zvazky zrazaju pod uhlom $\theta$.
\end{itemize}
Kineticku energiu v laboratornej sustave je mozne rozdelit na cast, ktora prislucha translacnemu pohybu sustavy castic - (kineticku energiu taziska) a cast, ktora prislucha relativnemu pohybu castic-(kineticka energia v taziskovej sustave). Kinematicke vztahy v taziskovej sustave sa vyznacuju maximalnou symetriou, co je jednoduchsie na vypocty. Vzhladom k tomu, ze vacsina experimentalnych vysledkov je ziskana v laboratornej sustave, je nutne medzi taziskovou a laboratornou sustavou prechadzat.\par
Zapiseme zakon zachovania hybnosti, zakon zachovania energie a rychlosti taziska v laboratornej sustave
\begin{equation}
\begin{gathered}
\vec{p_1}+\vec{p_2}=\vec{p_1}^{\prime}+\vec{p_2}^{\prime} \\
T_1+T_2=T_1^{\prime}+T_2^{\prime} \\
\vec{v}_T=\frac{m_1\vec{v}_1+m_2\vec{v}_2}{m_1+m_2}.
\end{gathered}
\end{equation}
Teraz uvedieme zakony zachovania a ryclost taziska v taziskovej sustave
\begin{equation}
\begin{gathered}
\tilde{\vec{p}}_1+\tilde{\vec{p}}_2=\tilde{\vec{p}}_1^{\prime}+\tilde{\vec{p}}_2^{\prime}=0 \\
\tilde{T}_1+\tilde{T}_2=\tilde{T}_1^{\prime}+\tilde{T}_2^{\prime} \\
\vec{v}_T=0.
\end{gathered}
\end{equation}
z coho plynie $$ \lvert \tilde{\vec{p}}_1 \rvert=\lvert \tilde{\vec{p}}_2 \rvert=\lvert \tilde{\vec{p}}_1 \rvert=\lvert \tilde{\vec{p}}_2  \rvert, \hspace{0.3cm} \tilde{\vec{v}}_1=\tilde{\vec{v}}_1^{\prime}, \hspace{0.3cm} \tilde{\vec{v}}_2=\tilde{\vec{v}}_2^{\prime}, \hspace{0.3cm} \tilde{T}_1=\tilde{T}_1^{\prime}, \hspace{0.3 cm} \tilde{T}_2=\tilde{T}_2^{\prime}. $$
Za predpokladu, ze je v laboratornej sustave tercikova castica v klidu ($p_2=0, T_2=0$), potom plati
\begin{equation}
\begin{gathered}
\tilde{\vec{v}}_1=\vec{v}_1-\vec{v}_T=\vec{v}_1-\frac{m_1\vec{v}_1+m_2\vec{v}_2}{m_1+m_2}=\frac{m_2}{m_1+m_2}\vec{v}_1\rightarrow \tilde{\vec{p}}_1=\mu \vec{v}_1 \\
\tilde{\vec{v}}_2=\vec{v}_2-\vec{v}_T=\vec{v}_2-\frac{m_1\vec{v}_1+m_2\vec{v}_2}{m_1+m_2}=-\frac{m_1}{m_1+m_2}\vec{v}_1\rightarrow \tilde{\vec{p}}_2=-\mu \vec{v}_1 \\
\tilde{T}=\tilde{T}_1+\tilde{T}_2=\frac{\tilde{p}_1^2}{2m_1}+\frac{\tilde{p}_2^2}{2m_2}=\frac{1}{2}\mu v_1^2=\frac{m_2}{m_1+m_2}T_1
\end{gathered}
\end{equation}
kde $\mu=\frac{m_1m_2}{m_1+m_2}$ je redukovana hmotnost. Toto su vztahy prechodu medzi taziskovou a laboratornou sustavou.

\section{Kinematické premenné}
Uvazujeme ze $c=\hbar=1$.\newline
Majme proces $a+b\rightarrow c+X$, kde $X$ su nespecifikovane castice. Casticu $c$ mozme povazovat za dcersku casticu od castice $a$ alebo $b$. Zrazku castic budeme uvazovat v sustave, kde zvazok casic $a$ nalietava v smere osi $z$ na terc tvoreny casticami $b$.\newline
Oznacme 
\begin{equation}
\begin{gathered}
p_a=(E_a,\vec{p}_{Ta},p_{za})\hspace{0.6cm} 4-impulz\hspace{0.2cm} nalietavajucej \hspace{0.2cm} castice \\
p_b=(E_b,\vec{p}_{Tb},p_{zb})\hspace{0.6cm} 4-impulz\hspace{0.2cm} tercikovej \hspace{0.2cm} castice,
\end{gathered}
\end{equation}
kde sme zaviedli tkz. priecnu hybnost, ktora je definovana ako $p_T=p\sin(\theta)$, kde $\theta$ je uhol rozptylu.\par
Hladame premenne, ktore su zlozene zo zloziek 4-impulzu castice (ktoru meriame) a maju nejaku specialnu vlastnost pri Lorentzovskej transformacii (co nam velmi zjednodusi popis). Pre detekovanu dcersku casticu $c$ definujeme
\begin{equation}
\begin{gathered}
c_+=E_c+p_{zc} \hspace{0.6cm} forward \hspace{0.2cm} lightcone \hspace{0.2cm} momentum \\
c_-=E_c-p_{zc} \hspace{0.6cm} backward \hspace{0.2cm} lightcone \hspace{0.2cm} momentum.
\end{gathered}
\end{equation}
Pre pomer dvoch lightcone premennych plati (zaroven to je Lorentzovky invariant)
$$ x_{\pm}=\frac{E_c\pm p_{cz}}{E_b\pm p_{bz}} \hspace{0.3cm} 0<x_{\pm}<1, $$ 
kde $x_{\pm}$ je forward (backward) lightcone premenna castice $c$ vzhladom k castici $b$.\par
Rovnako by sme mohli zavies $x_{\pm}$ vzhladom na casticu $a$ pretoze nie vzdy je mozne povedat, ktora castica je materskou casticou. Pokial skumame experiment, ktory produkuje castice v jednom preferovanom smere, berieme obvykle len jednu z lightcone premennych, druha sa nepouziva.\newline

\textbf{Rapidita}\par
Rapidita je bezrozměrná fyzikální veličina, která je mírou pohybu prostorem, podobně jako rychlost. Zatímco rychlost objektů je podle speciální teorie relativity shora omezena rychlostí světla ve vakuu $c$, rapidita může být libovolně velká. Pro objekty v klidu má hodnotu 0 a pro pomalé objekty je přímo úměrná rychlosti. Když se rychlost objektu přibližuje $c$, roste rapidita nade všechny meze.
Je definovana nasledovne $$ y=\frac{1}{2}\ln\bigg( \frac{E+p_z}{E-p_z} \bigg)=\frac{1}{2}\ln\bigg( \frac{x_+}{x_-} \bigg), $$
v nerelativisticke limite $y\rightarrow \beta$. Rapidita nie je lorentzovsky invariant, ale transformuje sa ako $\tilde{y}=y-y_{\beta}$ kde $y_{\beta}$ je rychlost pohybujucej sa vztaznej sustavy $S^{\prime}$.\par
Dalej plati
\begin{equation}
\begin{gathered}
E=m_T\cosh(y) \\
p_z=m_T\sinh(y),
\end{gathered}
\end{equation}
kde velicina $m_T$ je tzv. priecna hmotnost a je definovana nasledujucim sposobom
$$ m^2=E^2-p^2=E^2-p_z^2-p_T^2 \hspace{0.2cm} \rightarrow \hspace{0.2cm} E^2-p^2_z=m^2+p^2_T=m^2_T. $$\newline
\newpage
\textbf{Pseudorapidita}\par
Jej vyhodou je oproti rapidite v tom, ze staci jedna premenna pre jej definiciu - uhol vyletu. Pseudorapidita je definovana nasledovne
$$ \eta=-\ln\bigg( \tan\frac{\theta}{2}  \bigg) = \frac{1}{2}\ln \bigg( \frac{\lvert \vec{p} \rvert+p_z}{\lvert \vec{p} \rvert-p_z} \bigg), $$
kde uhol $\theta$ je uhol medzi hybnostou castice $\vec{p}$ a osou zvazku. V druhej casti vztahu mozme vidiet, ze pre velke hybnosti (resp. male hmotnosti) rapidita a pseudorapidita splyvaju ($p\approx E$).
\begin{figure}[!h]
\centering
\includegraphics[width=0.3\textwidth]{Pseudorapidity.png}
\caption{Tu mozme vidiet ako sa pseudorapidita meni s uhlom $\theta$.}
\label{sf1:fig:Pseudorapidity}
\end{figure}
\newline

\textbf{Feynmanova promenna}\par
Bola zavedena pri sudiu vysoko-energetickych zrazok hadronov pre popis elementarnej interakcie na kvarkovej urovni. Je definovana vztahom $$ x_F=\frac{\tilde{p}_z}{\tilde{p}_z^{max}}. $$
Feynmanova premenna je obvykle definovana v sustave, v ktorej sa castica pohybuje s nekonencou hybnostou (infinite momentum frame). Je tomu tak preto, lebo v kvantovej mechanike nie je operator poctu castic invariantny voci prechodu z jednej sustavy do druhej, a tak pocet castic, ktore pozorujeme pri lete vysoko-energetickej castice, zavisi na sustave, v ktorej proces studujeme. Limitna sustava je potom sustava, kde sa vsetky castice pohybuju s nekonecnou hybnostou, a tak doba zivota kvantovo vytvorenych castic je nekonecne mala a je tak mozne dobre definovat casticove obsadenie sustavy.\par
V tejto sustave mozeme ukazat,, ze $x_F=\frac{2\tilde{p}_z}{\sqrt{s}},$ dalej tiez plati $0<x_F<1$ a pre $E\rightarrow \infty$ mame $x_F=1$.\newline

\textbf{Bjorkenova promenna}\par
Jeto velicina, ktoa je definovana nasledovne $$ x=\frac{Q^2}{2(p_2\cdot q)}, $$ kde $Q^2=-q^2$. Na obrazku \ref{sf1:fig:Bjorken} su znazornene jednotlive 4-impulzy vyskytujuce sa v tejto premenne.
\begin{figure}[!h]
\centering
\includegraphics[width=0.6\textwidth]{Bjorken.png}
\caption{Rozptyl elektronu na protone.}
\label{sf1:fig:Bjorken}
\end{figure}
Podla obrazka mame
$$ p_4^2=M_W^2=(q+p_2)^2=q^2+2qp_2+p_2^2=-Q^2+2qp_2+M_p^2\rightarrow Q^2=2qp_2+M_p^2-M_W^2, $$
kedze je $M_p$ hmotnost protonu, ktory je najlahsi baryon, a $M_W$ hmotnost akehokolvek ineho baryonu tak $Q^2$ nebude nikd viac ako $2qp_2$.
Potom mame pre neelasticky rozptyl: $0<x<1$, zatial co pre elasticky rozptyl $x=1$.\par
($1-x$) ako keby reprezentovala cast prenesenej enegie, ktora sa spotrbuje na vytvorenie tazsieho baryonu (avsak toto je len moj nazor takze treba na to pozerat z nadhladom).

\end{document}